		\begin{enumerate}
			\item \begin{enumerate}
			    \item Computing the derivative of $f(x) = 1/\sqrt{x}$, we have $f'(x) = \frac{-1}{2\sqrt{x^3}}$ and $f''(x) = \frac{3}{4\sqrt{x^5}}$. 
                The first and second Taylor polynomials for $f$ are:
                \[p_1(x) = \frac{1}{2} -\frac{1}{16}x\]
                \[p_2(x) = \frac{1}{2} -\frac{1}{16}(x-4) + \frac{3}{128}(x-4)^2\]

                Comparing with a calculator, we have $|p_1(2) - f(2)| \approx 0.082$ and $|p_2(2) - f(2)| \approx 0.011$.

                \item The Lagrange polynomials for the points $3,4,5$ are:

                \[(x-4)(x-5)/(3-4)(3-5), (x-3)(x-5)/(4-3)(4-5), (x-3)(x-4)/(5-3)(5-4)\]
                Which amount to:
                \[\frac{(x-4)(x-5)}{2}, \frac{(x-3)(x-5)}{-1}, \frac{(x-3)(x-4)}{2}\]

                So the interpolating polynomial $q(x)$ through the points $(3,f(3)),(4,f(4)),(5,f(5))$ is:

                \[q(x) = \frac{1}{\sqrt{3}}\frac{(x-4)(x-5)}{2} + \frac{1}{\sqrt{4}}\frac{(x-3)(x-5)}{-1} + \frac{1}{\sqrt{5}}\frac{(x-3)(x-4)}{2}\]

                By direct evaluation we get $|q(2)-f(2)| \approx 0.0278$.

                \item Graphing the inequality $|f(x) - p_2(x)| < |f(x) - q(x)$ shows the region where the Taylor polynomial is better. Both approximations suffer as $x$ goes to zero. the Taylor approximation is better near $x=4$, but the interpolating polynomial is better for a larger region. 

                \item For this question we need the third and fourth derivatives of $f(x)$: $f'''(x) = \frac{-15}{8}x^{-7/2}$ and $f^{(4)}(x) = \frac{105}{16}x^{-9/2}$. With this, the third and fourth order Taylor polynomials are:

                \[p_3(x) = \frac{1}{2} -\frac{1}{16}(x-4) + \frac{3}{128}(x-4)^2 -\frac{15}{8\cdot 2^7} (x-4)^3\]
                \[p_4(x) = \frac{1}{2} -\frac{1}{16}(x-4) + \frac{3}{128}(x-4)^2 -\frac{15}{8}\cdot 4^{-7/2}(x-4)^3 + \frac{105}{16 \cdot 2^9}(x-4)^4\]

                Now we compute the errors at $0.5$ and $7.5$. The errors are $|p_3(2)-f(2)| \approx 0.128$ and $|p_4(2)-f(2)| \approx 0.333$. The error at $0.5$ is $|p_3(0.5)-f(0.5)| \approx 0.21$ and $|p_4(0.5)-f(0.5)| \approx 2.1$. The error at $7.5$ is $|p_3(7.5)-f(7.5)| \approx 0.42$ and $|p_4(7.5)-f(7.5)| \approx 1.49$. 

                \item As $x$ approaches $0$, the value of $f$ and its derivatives blows up. The remainder formula is:

                \[R_n(x) = f^{(n+1)}(c)/(n+1)!(x-a)^{n+1}\]

                Here's where the problem comes from: we don't know which $c$ this holds for, except that it lives in $(x,a)$. So to put the remainder formula to use we need an upper bound for $f^{(n+1)}(c)$. But because $\frac{1}{\sqrt{x}}$ has a singularity at zero, its derivatives are unbounded near zero! This suggests that no matter how many terms we have in our Taylor polynomial, the error will blow up near $x=0$. On the other hand, for large $x$ the derivative is bounded, but $(x-a)^{n+1}$ can become unwieldly. 

                
                
                

                
			\end{enumerate}
			\item \begin{enumerate}
			    \item 
                \begin{enumerate} \item 
                We'll need to know the first few derivatives of $L_k(x)$. A short computation tells us:

                \[L_k'(x) = \frac{2ke^{-2kx}}{(1+e^{-2kx})^2} = 2k\frac{1}{1+e^{-2kx}}(1-\frac{1}{1+e^{-2kx}})=2kL_k(x)(1-L_k(x))\]

                Notice that $L_k'(0) = k/2$. We can apply this identity a few times:

                \[L_k''(x) = L_k'(x)(1-L_k(x)) + L_k(x)(1-L_k'(x))\]

                Notice $L_k''(0) = 0$. 

                \[L_k'''(x) = L_k''(x)(1-L_k(x)) + 2L_k'(x)(1-L_k'(x)) + L_k(x)(1-L_k''(x))\]

                Since $L_k''(0) = 0$ and $L_k'(0) = k/2$, we get:

                \[L_k'''(0) = 2L_k'(0)(1-L_k'(0)) + L_k(0) = \frac{1+2k(1-k)}{2}\]

                So the Taylor series of $L_k$ based at zero is:

                \[p(x) = L_k(0) + L_k'(0)x +L_k'(0)x^2/2 + L_k'''(0)x^3/3!\]
                \item 
                
                \item We're looking to maximize $|H(x)-L_k(x)|$ over the interval $[\epsilon, 1]$. Since $x > \epsilon >0$, $|H(x) - L_k(x)| = 1 - L_k(x)$. The derivative of $L_k$ is:

                \[L_k'(x) = 2k\frac{1}{1+e^{-2kx}}(1-\frac{1}{1+e^{-2kx}})\]

                Which is always positive. It follows that $1-L_k(x)$ is always decreasing, and so obtains a maximum at $x = \epsilon$. 
                So for $x \in [\epsilon, 1]$:

                \[|L_k(x)-H(x)| \leq 1 -\frac{1}{1+e^{-2k\epsilon}} = \frac{e^{-2k\epsilon}}{1 + e^{-2k \epsilon}}\]

                We can see that the limit will go to zero. 
                
                \end{enumerate}


                
			\end{enumerate}
			\item \begin{enumerate}
				\item The required Taylor polynomials are:
                \[e^x = 1 + x + \frac{1}{2}x^2 + \frac{1}{6}x^3\]
                \[\cos{x} = 1 - \frac{1}{2}x^2 + \frac{1}{24}x^4 \]
                \[\sin{x} = x - \frac{1}{6}x^3\]
                \[\cosh{x} = 1 + \frac{1}{2}x^2 + \frac{1}{4}x^4\]
                \[\sinh{x} = x + \frac{1}{6}x^3\]

                \item The required Taylor polynomials are:
                \[e^{ix} = 1 + ix - \frac{1}{2}x^2 - \frac{1}{6}x^3 + \frac{1}{24}x^4\]
                \[\cos{x} + i\sin{x} = 1 - \frac{1}{2}x^2 + \frac{1}{24}x^4 + ix - \frac{i}{6}x^3\]

                This demonstrates Euler's identity where $e^{ix} = \cos{x} + i\sin{x}$.

                \item The required Taylor polynomial is:
                \[\cosh{x} + \sinh{x} = 1 + \frac{1}{2}x^2 + \frac{1}{4}x^4 + x + \frac{1}{6}x^3\]

                Looking at the Taylor polynomial for $e^x$ from Part A, we can see that the identity $e^x = \cosh{x} + \sinh{x}$ holds.

			\end{enumerate}
		\end{enumerate}
