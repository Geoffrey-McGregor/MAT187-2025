\documentclass[red]{tutorial}
\usepackage[no-math]{fontspec}
\usepackage{xpatch}
	\renewcommand{\ttdefault}{ul9}
	\xpatchcmd{\ttfamily}{\selectfont}{\fontencoding{T1}\selectfont}{}{}
	\DeclareTextCommand{\nobreakspace}{T1}{\leavevmode\nobreak\ }
\usepackage{polyglossia} % English please
	\setdefaultlanguage[variant=us]{english}
%\usepackage[charter,cal=cmcal]{mathdesign} %different font
%\usepackage{avant}
\usepackage{microtype} % Less badboxes

%\usepackage{enumitem}

\usepackage[charter,cal=cmcal]{mathdesign} %different font
%\usepackage{euler}
 
\usepackage{blindtext}
\usepackage{calc, ifthen, xparse, xspace}
\usepackage{makeidx}
\usepackage[hidelinks, urlcolor=blue]{hyperref}   % Internal hyperlinks
\usepackage{mathtools} % replaces amsmath
\usepackage{bbm} %lower case blackboard font
\usepackage{amsthm, bm}
\usepackage{thmtools} % be able to repeat a theorem
\usepackage{thm-restate}
\usepackage{graphicx}
\usepackage[dvipsnames]{xcolor}
\usepackage{multicol}
\usepackage{fnpct} % fancy footnote spacing
\usepackage{tikz}
\usetikzlibrary{arrows.meta}

\usepackage{pgfplots}
\pgfplotsset{compat=1.18}
%\pgfkeys{/pgf/fpu}

 \usepackage{enumitem}
 
\newcommand{\xh}{{{\mathbf e}_1}}
\newcommand{\yh}{{{\mathbf e}_2}}
\newcommand{\zh}{{{\mathbf e}_3}}
\newcommand{\R}{\mathbb{R}}
\newcommand{\Z}{\mathbb{Z}}
\newcommand{\N}{\mathbb{N}}
\newcommand{\proj}{\mathrm{proj}}
\newcommand{\Proj}{\mathrm{proj}}
\newcommand{\Perp}{\mathrm{perp}}
\renewcommand{\span}{\mathrm{span}\,}
\newcommand{\Span}{\mathrm{span}\,}
\newcommand{\Img}{\mathrm{img}\,}
\newcommand{\Null}{\mathrm{null}\,}
\newcommand{\Range}{\mathrm{range}\,}
\newcommand{\rref}{\mathrm{rref}}
\newcommand{\rank}{\mathrm{rank}}
\newcommand{\Rank}{\mathrm{rank}}
\newcommand{\nnul}{\mathrm{nullity}}
\newcommand{\mat}[1]{\begin{bmatrix}#1\end{bmatrix}}
\newcommand{\chr}{\mathrm{char}}
\renewcommand{\d}{\mathrm{d}}


\theoremstyle{definition}
\newtheorem{example}{Example}[section]
\newtheorem{defn}{Definition}[section]

%\theoremstyle{theorem}
\newtheorem{thm}{Theorem}[section]

\pgfkeys{/tutorial,
	name={Tutorial 9},
	author={},
	course={MAT 187},
	date={},
	term={},
	title={Applications of ODEs}
	}

\begin{document}
	\begin{tutorial}
		\begin{objectives}
	In this tutorial you will learn how to leverage Euler's method to approximate the solution of boundary-value problems.

	This tutorial relates to the following course learning objectives:
	\textit{
		Use a computer to approximate the solutions to differential equations and systems of differential
		equations and explain the advantages and draw-backs of computer-based approximations
	}.
\end{objectives}


\vspace{-.5em}
\subsection*{Problems}
\vspace{-.5em}




%%%%%%%%%%%%%%%%%%%%%%%%%%


\begin{enumerate}
	\item\label{QIVP} Consider the initial-value problem:
	      \begin{equation}\tag{IVP}\label{IVP}
		      \begin{cases}
			      w''(t) = -w(t) \\
			      w(0) = 0       \\
			      w'(0) = 5
		      \end{cases}
	      \end{equation}


	      \begin{enumerate}
		      \item Let $\vec{r}(t) = \mat{w \\ w'}$. 
			  Write the system of differential equations above in matrix form (i.e., as $\vec{r}'=M \vec{r}$). 
			  What is the initial condition for $\vec{r}$?
		      \item Use a spreadsheet to approximate the value of $\vec{r}(1)$ (use a step size $\leq 0.01$).
		      \item Find an approximate value for $w(1)$.
	      \end{enumerate}



	\item Consider the boundary-value problem:
	      \begin{equation}\tag{BVP}\label{BVP}
		      \begin{cases}
			      w''(t) = -w(t) \\
			      w(0) = 0       \\
			      w(1) = 1
		      \end{cases}
	      \end{equation}


	      %	We want to approximate the solution of this differential equation with these two boundary conditions \eqref{BVP}.

	      In Question \ref{QIVP}, we approximated
		   a second-order differential equation with initial conditions by applying Euler's method to a corresponding system.
		   We will modify this approach to approximate the solution to \eqref{BVP}.

	      This method is called the \textbf{Shooting Method}.

	      Notation: \textit{In this question, $w_a(t)$ will denote the Euler approximation to Equations 
		  \eqref{IVP} with $w'(0)=a$.}


	      \begin{enumerate}
		      \item Find a number $a$ such that if $w'(0)=a$, then the Euler approximation satisfies $w_{a}(1)<1$.

		      \item Find a number $b$ such that If $w'(0)=b$, then the Euler approximation satisfies $w_{b}(1)>1$.

		      \item Let $c = \frac{a+b}{2}$. Use Euler's Method again to approximate $w_c(1)$.

			  \item If the boundary problem has a solution, there is a value $k$ so that $w_k(1)=1$.
			  Based on your previous results, is $k\in [a,c]$ or is $k\in [c,b]$? Explain.

		      \item We will try to find $k$ by using \emph{bisection}. That is, when we identify an interval
			  $[a_i, b_i]$ that must contain $k$, we will test $w_{c_i}(1)$ where $c_i=\frac{a_i+b_i}{2}$ and then
			  narrow our search to either $[a_i, c_i]$ or $[c_i,b_i]$.

		      Using $a_1=a$, $b_1=b$, and $c_1=c$ (which you calculated already), fill out the following table:

		            \begin{tabular}{|c||c|c||c|c||c|c|}
			            \hline
			            $i$ & \hspace{20pt} $a_i$ \hspace{20pt} & \hspace{10pt} $w_{a_i}(1)$  \hspace{10pt} & $b_i$  \hspace{20pt} & \hspace{10pt} $w_{b_n}(1)$ \hspace{10pt} & \hspace{20pt} $c_i$  \hspace{20pt} & \hspace{10pt} $w_{c_i}(1)$ \hspace{10pt} \\ \hline
			            1   &                                   &                                           &                      &                                          &                                    &                                          \\[10pt]\hline
			            2   &                                   &                                           &                      &                                          &                                    &                                          \\[10pt]\hline
			            3   &                                   &                                           &                      &                                          &                                    &                                          \\[10pt]\hline
			            4   &                                   &                                           &                      &                                          &                                    &                                          \\[10pt]\hline
			            5   &                                   &                                           &                      &                                          &                                    &                                          \\[10pt]\hline
			            6   &                                   &                                           &                      &                                          &                                    &                                          \\[10pt]\hline
			            7   &                                   &                                           &                      &                                          &                                    &                                          \\[10pt]\hline
		            \end{tabular}
		      \item The iterative process you just performed is called the \emph{Shooting Method}\footnote{ In this case,
			  you used bisection to narrow down the ``target''; the shooting method does not require bisection. For example,
			  you could use Newton's method to narrow the target.}.
			  
			  Use the following Desmos plot to compare the solution you arrived at via the Shooting Method 
			  with the exact solution.
			  
			 \url{https://www.desmos.com/calculator/mvjr18tt2x}
	      \end{enumerate}


	\item The BVP below has infinitely many solutions.
	      \begin{equation}\tag{$\star$}\label{BVP2}
		      \begin{cases}
			      w''(t) = -\pi^2 \cdot w(t) + \cos\Big(w(t)\Big) \\
			      w(0) = 0                   \\
			      w(1) = 0
		      \end{cases}
	      \end{equation}

	      Use the Shooting Method to identify at least \emph{two} solutions to this boundary value problem.

		  Hint: You may need to pick very different initial values for $w'(0)$ to find intervals that contain different solutions.

\end{enumerate}



%
%
%\vfill
%
%
%
%\paragraph{Shooting Method:}	 To approximate the solution of the boundary-value problem:
%$$
%	\begin{cases}
%		w'' = F\big(t,w,\tfrac{dw}{dt}\big) \\
%		w(t_0)=w_0                          \\
%		w(t_1)=w_1
%	\end{cases}
%$$
%\begin{enumerate}[label={(Step \arabic*)}]
%	\item Find two values $a$ and $b$ such that:
%	      \begin{itemize}
%		      \item If $w'(t_0)=a$, then the Euler approximation satisfies $w_a(t_1)<w_1$
%		      \item If $w'(t_0)=b$, then the Euler approximation satisfies $w_b(t_1)>w_1$
%	      \end{itemize}
%
%	\item \label{step}Let $c = \frac{a+b}{2}$. With $w'(t_0)=c$, use Euler's method to obtain an approximation of $w_c(t_1)$.
%
%	\item Repeat \ref{step} with new values for $(a,c)$ or $(b,c)$ instead of $(a,b)$, making sure that $w_1 \in [w_a(t_1), w_b(t_1)]$.
%\end{enumerate}















%%%%%%%%%%%%%%%%%%%%%%%%%%






	\end{tutorial}

	\begin{solutions}
		

\begin{enumerate}
	\item \begin{enumerate}
		      \item $\vec r\,' = \mat{0&1\\-1&0}\vec r$. The initial condition is $\vec r(0) = \mat{0\\5}$.
		      \item With $\Delta=0.01$ we have $\vec r(1)\approx\mat{4.228\\2.715}$.
		      \item With $\Delta=0.01$ we have $w(1)\approx4.228$.
	      \end{enumerate}
	\item \begin{enumerate}
		      \item We can pick $a=0$
		      \item We can pick $b=2$
		      \item With $c=1$, we have $w_c(1)\approx 0.846$.
		      \item $k\in [c,b]$ based on our values of $a$, $b$, and $c$. This makes sense because
		            the value of $w_x(1)$ should depend continuously on $x$, so by the intermediate value theorem,
		            there must be a $k\in [c,b]$ so that $w_k(1)=1$. (Though, this does not rule out there also
		            being such a $k$ in $[a,b]$.)
		      \item \phantom{x}

		            \begin{tabular}{|c||c|c||c|c||c|c|}
			            \hline
			            $i$ & \hspace{20pt} $a_i$ \hspace{20pt} & \hspace{10pt} $w_{a_i}(1)$  \hspace{10pt} & $b_i$  \hspace{20pt} & \hspace{10pt} $w_{b_n}(1)$ \hspace{10pt} & \hspace{20pt} $c_i$  \hspace{20pt} & \hspace{10pt} $w_{c_i}(1)$ \hspace{10pt} \\ \hline
			            1   & 0                                 & 0                                         & 2                    & 1.691                                    & 1                                  & 0.846                                    \\[10pt]\hline
			            2   & 1                                 & 0.846                                     & 2                    & 1.691                                    & 1.5                                & 1.269                                    \\[10pt]\hline
			            3   & 1                                 & 0.846                                     & 1.5                  & 1.269                                    & 1.25                               & 1.057                                    \\[10pt]\hline
			            4   & 1                                 & 0.846                                     & 1.25                 & 1.057                                    & 1.125                              & 0.951                                    \\[10pt]\hline
			            5   & 1.125                             & 0.951                                     & 1.25                 & 1.057                                    & 1.1875                             & 1.004                                    \\[10pt]\hline
			            6   & 1.125                             & 0.951                                     & 1.1875               & 1.004                                    & 1.15625                            & 0.978                                    \\[10pt]\hline
			            7   & 1.15625                           & 0.978                                     & 1.1875               & 1.004                                    & 1.171875                           & 0.991                                    \\[10pt]\hline
		            \end{tabular}
                \item 
	      \end{enumerate}
	\item Using the shooting method we find solutions to the boundary value problem with $w'(0)\approx 6.07$ and $w'(0)\approx 17.25$
    (using a step size of $\Delta=0.001$).
\end{enumerate}
	
	\end{solutions}
	\begin{instructions}
		\subsection*{Learning Objectives}
Students need to be able to\ldots
\begin{itemize}
	\item Build a spreadsheet that can use Euler's method to approximate the solution to a system of differential equations
	\item Rewrite a higher-order ODE as a system of ODEs.
	\item Use Euler's method to simulate the solution to a higher-order ODE.
	\item Numerically approximate the solution to a boundary value problem involving higher-order ODEs.
\end{itemize}

\subsection*{Context}

In class we used Euler's method extensively.

We also introduced boundary-value problems (BVPs) in class to show that studying existence 
and uniqueness of solution is not a trivial matter.

For this type of differential equation, however, we can't use Euler's method directly, since it requires initial conditions.
Instead, we repeatedly apply Euler's method to find better and better guesses at the initial conditions.


\subsection*{Resources for TAs}

Excel spreadsheet with the Shooting Method.

\begin{itemize}
	\item \url{https://utoronto-my.sharepoint.com/:x:/g/personal/bernardo_galvao_sousa_utoronto_ca/EQO08o4-PdFMr8PMd80LmP0BPilrLuGIcNMpP-oVpfM4ow?e=IR39N4}
\end{itemize}




\subsection*{What to Do}
Introduce the learning objectives for the day's tutorial.

Explain the big picture: if we can rewrite a higher-order ODE in terms of a system of first-order ODEs, 
we can use Euler's method to approximate the solution to the higher-order ODE.

Have students get into small groups and start on \#1. Each group needs to have at least 1 laptop.
This question should be done quickly---it's setting the stage for \#2.




\subsection*{Notes}

	\begin{enumerate}
		\item This should be a quick review.
		\item This question carefully walks students through the Shooting Method. If students get stuck on
		this one, go ahead and give them a big-picture overview of the Shooting Method.
		\item This is here for the quicker students, though they may need a smaller step size than $\Delta=0.01$ to get good results.
	\end{enumerate}






	\end{instructions}

\end{document}
