\documentclass[red]{tutorial}
\usepackage[no-math]{fontspec}
\usepackage{xpatch}
	\renewcommand{\ttdefault}{ul9}
	\xpatchcmd{\ttfamily}{\selectfont}{\fontencoding{T1}\selectfont}{}{}
	\DeclareTextCommand{\nobreakspace}{T1}{\leavevmode\nobreak\ }
\usepackage{polyglossia} % English please
	\setdefaultlanguage[variant=us]{english}
%\usepackage[charter,cal=cmcal]{mathdesign} %different font
%\usepackage{avant}
\usepackage{microtype} % Less badboxes

%\usepackage{enumitem}

\usepackage[charter,cal=cmcal]{mathdesign} %different font
%\usepackage{euler}
 
\usepackage{blindtext}
\usepackage{calc, ifthen, xparse, xspace}
\usepackage{makeidx}
\usepackage[hidelinks, urlcolor=blue]{hyperref}   % Internal hyperlinks
\usepackage{mathtools} % replaces amsmath
\usepackage{bbm} %lower case blackboard font
\usepackage{amsthm, bm}
\usepackage{thmtools} % be able to repeat a theorem
\usepackage{thm-restate}
\usepackage{graphicx}
\usepackage[dvipsnames]{xcolor}
\usepackage{multicol}
\usepackage{fnpct} % fancy footnote spacing
\usepackage{tikz}
\usetikzlibrary{arrows.meta}

\usepackage{pgfplots}
\pgfplotsset{compat=1.18}
%\pgfkeys{/pgf/fpu}

 
\newcommand{\xh}{{{\mathbf e}_1}}
\newcommand{\yh}{{{\mathbf e}_2}}
\newcommand{\zh}{{{\mathbf e}_3}}
\newcommand{\R}{\mathbb{R}}
\newcommand{\Z}{\mathbb{Z}}
\newcommand{\N}{\mathbb{N}}
\newcommand{\proj}{\mathrm{proj}}
\newcommand{\Proj}{\mathrm{proj}}
\newcommand{\Perp}{\mathrm{perp}}
\renewcommand{\span}{\mathrm{span}\,}
\newcommand{\Span}{\mathrm{span}\,}
\newcommand{\Img}{\mathrm{img}\,}
\newcommand{\Null}{\mathrm{null}\,}
\newcommand{\Range}{\mathrm{range}\,}
\newcommand{\rref}{\mathrm{rref}}
\newcommand{\rank}{\mathrm{rank}}
\newcommand{\Rank}{\mathrm{rank}}
\newcommand{\nnul}{\mathrm{nullity}}
\newcommand{\mat}[1]{\begin{bmatrix}#1\end{bmatrix}}
\newcommand{\chr}{\mathrm{char}}
\renewcommand{\d}{\mathrm{d}}


\theoremstyle{definition}
\newtheorem{example}{Example}[section]
\newtheorem{defn}{Definition}[section]

%\theoremstyle{theorem}
\newtheorem{thm}{Theorem}[section]

\pgfkeys{/tutorial,
	name={Tutorial 8},
	author={},
	course={MAT 187},
	date={},
	term={},
	title={Linear ODEs}
	}

\begin{document}
	\begin{tutorial}
		\begin{objectives}
	In this tutorial you explore the limits of what affine approximations can
	tell you about the nature of equilibrium solutions.

	These problems relate to the following course learning objectives:
	\textit{Apply linear algebra techniques to classify solutions of linear systems of ordinary differential
		equations including rigorously classifying the stability of equilibrium solutions and creating
		linear approximations to non-linear systems of ordinary differential equations}.
\end{objectives}

\subsection*{Problems}

Linearization is a powerful technique to classify critical points, but it can provide inconclusive results.

\begin{enumerate}
	\item Consider the differential equations
	      \begin{equation}
		      y'=-y^3-y \tag{A}
	      \end{equation}
	      \begin{equation}
		      y'=-y^3 \tag{B}
	      \end{equation}
	      \begin{equation}
		      y'=y^3 \tag{C}
	      \end{equation}
	      \begin{enumerate}
		      \item Use Desmos to graph the slope field for each equation. Based on the slope
		            field, how would you classify the equilibrium solution $y=0$ for each equation?

		            \url{https://www.desmos.com/calculator/ghavqzqqjn}
		            %\item Find the general solution to each differential equation and use the general solution to classify the equilibrium $y(t)=0$
		            %      as stable/unstable.
		      \item Find affine approximations for each differential equation centered at the equilibrium $y=0$.
		      \item Classify the critical points of each of your affine approximations as stable/unstable.
		      \item Why didn't your affine approximations correctly predict stability/instability in all cases? Explain.
		      \item In calculus, the \emph{first derivative test} states that at a point:
		            (i) if the derivative is positive, the function is increasing; (ii)
		            if				  the derivative is negative, the function is decreasing; and
		            (iii) if the derivative is zero, the test is inconclusive.

		            Make up your own ``affine approximation test'' for the stability/instability of an equilibrium solution
		            for a differential equation.
	      \end{enumerate}

	\item Let $\vec r(t)=\mat{x(t)\\y(t)}$ and consider the differential equation
	      \begin{equation}
		      \vec r\,' = \mat{-y-x^3\\x-y^3} \tag{D}
	      \end{equation}
	      which has $\vec r(t)=(0,0)$ as its only equilibrium solution.
	      \begin{enumerate}
		      \item Find an affine approximation Equation (D) centered at $(0,0)$.
		      \item Classify the nature of the equilibrium solution in your affine approximation.
		      \item Do you believe that the nature of the equilibrium solution in your affine approximation
		            is the same as the nature of the equilibrium solution in Equation (D)? Why or why not?
		      \item Make a phase portrait for Equation (D). Can you tell what the nature of the equilibrium solution is?

		            \textbf{Note:} Make sure to zoom in on your phase portrait before you draw a strong conclusion.
		      \item Use numerical simulations to classify the nature of the equilibrium solution.

		            \textbf{Hint:} You may need to use \emph{very} small step sizes to ensure that rounding error
		            doesn't obscure your results.

		            %\item Come up with an analytic argument to classify the nature of the equilibrium solution.

		      \item Suppose you have an affine approximation of a system of differential equations. Further suppose
		            that the matrix in your affine approximation has eigenvalues $a_1+b_1i$, $a_2+b_2i$, $a_3+b_3i$, \ldots, $a_n+b_ni$.
		            Explain how to classify an equilibrium solution as stable/unstable based on the eigenvalues.

	      \end{enumerate}
\end{enumerate}
	\end{tutorial}

	\begin{solutions}
		\begin{enumerate}
	\item
	      \begin{enumerate}
		      \item
		            \textbf{Equation (A)}: Stable and attracting.

		            \textbf{Equation (B)}: Stable and attracting.

		            \textbf{Equation (C)}: Unstable and repelling.

				\item Equation (A) has an affine approximation $y'=-y$. Equations (B) and (C) 
					have same affine approximation: $y'=0$.
		      \item Every affine approximation has a stable equilibrium solutions at $0$. The affine approximation for Equation (A) is
			  	also attracting.
		      \item Affine approximations are based off of the first derivative. For Equations (B) and (C), the first
		            derivative at the origin is zero, so it fails to differentiate between cases.
		      \item Given an autonomous equation $y'=f(y)$ with an equilibrium at $y=k$, the equilibrium
		            is attracting if $f'(k) < 0$, repelling if $f'(k) > 0$ and more investigation is needed if $f'(k) = 0$.
	      \end{enumerate}

	\item
	      \begin{enumerate}
		      \item $\vec r\,' = \mat{0&-1\\1&0}\vec r$.
		      \item The equilibrium for the affine approximation is stable (not repelling nor attracting).
		      \item No. Similar to Question 1, there are cubics that the first derivative fails to pick up on. These
		            cubics surely cause problems!
		      \item It looks like solutions circle about the origin, but it is hard to tell much beyond that.
		      \item Based on numerical simulations using Euler's method with $\Delta <0.01$, it appears that solutions
		            very slowly circle inwards.
		      %\item If we consider the differential equations $\vec r\,' =\mat{-y\\x}$ and $\vec r\,'=\mat{-x^3\\-y^3}$
		      %      in isolation, the first one has periodic solutions that circle about the origin and the second one
		      %      has solutions that head straight towards the origin. Equation (D) is a sum of these two equations, so
		      %      it makes sense that solutions circle but also tend towards the origin.

		      %      We can quantify how much solutions move towards the origin. We know $\vec r\,'(x,y)$ is a tangent
		      %      vector to a solution curve at the point $(x,y)$ and that the vector $(x,y)$ points radially out
		      %      from the equilibrium solution to the point $(x,y)$. Thus, computing
		      %      \[
			  %          \vec r\,'\cdot \mat{x\\y}=\mat{-y-x^3\\x-y^3}\cdot \mat{x\\y}=-(x^4+y^4)
		      %      \]
		      %      we see that the angle between $\vec r\,'$ and $(x,y)$ is always greater than $90^\circ$. Thus, solution
		      %      curves tend slightly towards the origin, making the equilibrium attracting.

		      \item If the real part of all eigenvalues is non-zero we can use the eigenvalues to classify the equilibrium
		            as follows. If all real parts are positive, the equilibrium is repelling and unstable. If all real parts are negative, the equilibrium
		            is attracting and stable. If there is a mix of positive and negative real parts, the equilibrium is unstable.

		            Otherwise, if at least one real part is positive, we know the equilibrium is unstable (but cannot determine whether it is repelling).
		            If at least one real part is positive and one real part is negative, we know the equilibrium is unstable
		            and not repelling.

		            In all other cases, we cannot conclude from the eigenvalues the nature of the equilibrium.
	      \end{enumerate}
\end{enumerate}
	
	\end{solutions}
	\begin{instructions}
		\subsection*{Learning Objectives}
Students need to be able to\ldots
\begin{itemize}
	\item Find affine approximations to a differential equation or a system of differential equations centered
	      at an equilibrium solution.
	\item Explain why an equilibrium solution may be unstable even if the equilibrium solution
	      for the corresponding affine approximation is stable.
\end{itemize}

\subsection*{Context}

In class we have only used affine approximations to classify equilibrium solutions when the
nature of the equilibrium solution is the same for the differential equation and the affine approximation.
Since this is not always the case, we need to train students how to tell if their affine approximation
gives the required information.


\subsection*{What to Do}

This tutorial is as usual. Start by stating the learning objectives for the day. Then have students
get into groups and start on the problems. Walk around and help encourage groups who are stuck.
6 minutes before the end of tutorial, pick a suitable problem to do as a wrap-up.

\subsection*{Notes}
\begin{enumerate}
	\item 
	\begin{enumerate}
		\item It should be obvious, but students may need to ``shrink the window'' to see details of
		the slope field near the origin.
		\item This may take them some time. If they are stuck tell them to check their work from last tutorial.
		\item We mostly do this in 2d. They should know how to do it in 1d, though. For some of the students,
		telling them to think of $1\times 1$ matrices will help (for others that will just confuse them).
		\item 
		\item Have them actually write down their test. This will help them clarify their ideas.
	\end{enumerate}
	\item
	\begin{enumerate}
		\item This may take them some time, but they need to be able to do this for the exams.
		\item
		\item 
		\item 
		\item They may need step sizes smaller than $\Delta=0.01$ to get a reasonable approximation.
		The closer to the origin they are, the smaller the needed step size is.
		\item Encourage them to make a table or list to enumerate all the options.
	\end{enumerate}
\end{enumerate}
	\end{instructions}

\end{document}
