\documentclass[red]{tutorial}
\usepackage[no-math]{fontspec}
\usepackage{xpatch}
	\renewcommand{\ttdefault}{ul9}
	\xpatchcmd{\ttfamily}{\selectfont}{\fontencoding{T1}\selectfont}{}{}
	\DeclareTextCommand{\nobreakspace}{T1}{\leavevmode\nobreak\ }
\usepackage{polyglossia} % English please
	\setdefaultlanguage[variant=us]{english}
%\usepackage[charter,cal=cmcal]{mathdesign} %different font
%\usepackage{avant}
\usepackage{microtype} % Less badboxes


\usepackage[charter,cal=cmcal]{mathdesign} %different font
%\usepackage{euler}
 
\usepackage{blindtext}
\usepackage{calc, ifthen, xparse, xspace}
\usepackage{makeidx}
\usepackage[hidelinks, urlcolor=blue]{hyperref}   % Internal hyperlinks
\usepackage{mathtools} % replaces amsmath
\usepackage{bbm} %lower case blackboard font
\usepackage{amsthm, bm}
\usepackage{thmtools} % be able to repeat a theorem
\usepackage{thm-restate}
\usepackage{graphicx}
\usepackage{xcolor}
\usepackage{multicol}
\usepackage{fnpct} % fancy footnote spacing

 
\newcommand{\xh}{{{\mathbf e}_1}}
\newcommand{\yh}{{{\mathbf e}_2}}
\newcommand{\zh}{{{\mathbf e}_3}}
\newcommand{\R}{\mathbb{R}}
\newcommand{\Z}{\mathbb{Z}}
\newcommand{\N}{\mathbb{N}}
\newcommand{\proj}{\mathrm{proj}}
\newcommand{\Proj}{\mathrm{proj}}
\newcommand{\Perp}{\mathrm{perp}}
\renewcommand{\span}{\mathrm{span}\,}
\newcommand{\Span}{\mathrm{span}\,}
\newcommand{\Img}{\mathrm{img}\,}
\newcommand{\Null}{\mathrm{null}\,}
\newcommand{\Range}{\mathrm{range}\,}
\newcommand{\rref}{\mathrm{rref}}
\newcommand{\rank}{\mathrm{rank}}
\newcommand{\Rank}{\mathrm{rank}}
\newcommand{\nnul}{\mathrm{nullity}}
\newcommand{\mat}[1]{\begin{bmatrix}#1\end{bmatrix}}
\newcommand{\chr}{\mathrm{char}}
\renewcommand{\d}{\mathrm{d}}


\theoremstyle{definition}
\newtheorem{example}{Example}[section]
\newtheorem{defn}{Definition}[section]

%\theoremstyle{theorem}
\newtheorem{thm}{Theorem}[section]

\pgfkeys{/tutorial,
	name={Tutorial 1},
	author={Jason Siefken \& Bernardo Galv\~ao-Sousa},
	course={MAT 244},
	date={},
	term={},
	title={Limits of Numerics}
	}

\begin{document}
	\begin{tutorial}
				\begin{objectives}
			In this tutorial, you practice creating and using polynomial approximations of
			functions.
		\end{objectives}

	%	\bigskip


		\subsection*{Problems}
		
		\begin{enumerate}
			\item  The inverse square root function, $f(x) = 1/\sqrt{x}$ is an important function
			when simulating light and reflections in computer graphics.  However, computing square 
			roots can be expensive, so we will use much-simpler polynomial approximations.
				\begin{enumerate}
					\item Find the first and second degree Taylor approximations to $f$ at $x=4$.
					Use each of these polynomials to approximate $f(2)$. Use a calculator 
					to quantify the error in each approximation.
					\item Find the quadratic polynomial that interpolates between the points 
					$(3,f(3))$, $(4,f(4))$, and $(5,f(5))$. 
					Use this polynomial to approximate $f(2)$ and quantify its error.

					\item Use Desmos to graphically answer the following: 
					In what region(s) is the interpolating polynomial better?
					In what region(s) is the Taylor polynomial better?

					\item Find the third and fourth degree Taylor approximations to $f$ at $x=4$.
					How well do they approximate $f(0.5)$. What about $f(7.5)$?

					\item % Finish off with insights about the asymptote blocking convergence
				\end{enumerate}

				\item % Try to approximate a step function? Let them use either
				% interpolation or the Taylor series of something that becomes just like the 
				% step function? This could be fund and frustrating!

				\item % A question where they need to approximate a function on R^2 by
				% using a complex polynomial and then taking the real and imaginary parts
				% would be neat.
		\end{enumerate}

	\end{tutorial}

	\begin{solutions}
				\begin{enumerate}
			\item \begin{enumerate}
			    \item Computing the derivative of $f(x) = 1/\sqrt{x}$, we have $f'(x) = \frac{-1}{2\sqrt{x^3}}$ and $f''(x) = \frac{3}{4\sqrt{x^5}}$. 
                The first and second Taylor polynomials for $f$ are:
                \[p_1(x) = \frac{1}{2} -\frac{1}{16}x\]
                \[p_2(x) = \frac{1}{2} -\frac{1}{16}(x-4) + \frac{3}{128}(x-4)^2\]

                Comparing with a calculator, we have $|p_1(2) - f(2)| \approx 0.082$ and $|p_2(2) - f(2)| \approx 0.011$.

                \item The Lagrange polynomials for the points $3,4,5$ are:

                \[(x-4)(x-5)/(3-4)(3-5), (x-3)(x-5)/(4-3)(4-5), (x-3)(x-4)/(5-3)(5-4)\]
                Which amount to:
                \[\frac{(x-4)(x-5)}{2}, \frac{(x-3)(x-5)}{-1}, \frac{(x-3)(x-4)}{2}\]

                So the interpolating polynomial $q(x)$ through the points $(3,f(3)),(4,f(4)),(5,f(5))$ is:

                \[q(x) = \frac{1}{\sqrt{3}}\frac{(x-4)(x-5)}{2} + \frac{1}{\sqrt{4}}\frac{(x-3)(x-5)}{-1} + \frac{1}{\sqrt{5}}\frac{(x-3)(x-4)}{2}\]

                By direct evaluation we get $|q(2)-f(2)| \approx 0.0278$.

                \item Graphing the inequality $|f(x) - p_2(x)| < |f(x) - q(x)$ shows the region where the Taylor polynomial is better. Both approximations suffer as $x$ goes to zero. the Taylor approximation is better near $x=4$, but the interpolating polynomial is better for a larger region. 

                \item For this question we need the third and fourth derivatives of $f(x)$: $f'''(x) = \frac{-15}{8}x^{-7/2}$ and $f^{(4)}(x) = \frac{105}{16}x^{-9/2}$. With this, the third and fourth order Taylor polynomials are:

                \[p_3(x) = \frac{1}{2} -\frac{1}{16}(x-4) + \frac{3}{128}(x-4)^2 -\frac{15}{8\cdot 2^7} (x-4)^3\]
                \[p_4(x) = \frac{1}{2} -\frac{1}{16}(x-4) + \frac{3}{128}(x-4)^2 -\frac{15}{8}\cdot 4^{-7/2}(x-4)^3 + \frac{105}{16 \cdot 2^9}(x-4)^4\]

                Now we compute the errors at $0.5$ and $7.5$. The errors are $|p_3(2)-f(2)| \approx 0.128$ and $|p_4(2)-f(2)| \approx 0.333$. The error at $0.5$ is $|p_3(0.5)-f(0.5)| \approx 0.21$ and $|p_4(0.5)-f(0.5)| \approx 2.1$. The error at $7.5$ is $|p_3(7.5)-f(7.5)| \approx 0.42$ and $|p_4(7.5)-f(7.5)| \approx 1.49$. 

                \item As $x$ approaches $0$, the value of $f$ and its derivatives blows up. The remainder formula is:

                \[R_n(x) = f^{(n+1)}(c)/(n+1)!(x-a)^{n+1}\]

                Here's where the problem comes from: we don't know which $c$ this holds for, except that it lives in $(x,a)$. So to put the remainder formula to use we need an upper bound for $f^{(n+1)}(c)$. But because $\frac{1}{\sqrt{x}}$ has a singularity at zero, its derivatives are unbounded near zero! This suggests that no matter how many terms we have in our Taylor polynomial, the error will blow up near $x=0$. On the other hand, for large $x$ the derivative is bounded, but $(x-a)^{n+1}$ can become unwieldly. 

                
                
                

                
			\end{enumerate}
			\item \begin{enumerate}
			    \item 
                \begin{enumerate} \item 
                We'll need to know the first few derivatives of $L_k(x)$. A short computation tells us:

                \[L_k'(x) = \frac{2ke^{-2kx}}{(1+e^{-2kx})^2} = 2k\frac{1}{1+e^{-2kx}}(1-\frac{1}{1+e^{-2kx}})=2kL_k(x)(1-L_k(x))\]

                Notice that $L_k'(0) = k/2$. We can apply this identity a few times:

                \[L_k''(x) = L_k'(x)(1-L_k(x)) + L_k(x)(1-L_k'(x))\]

                Notice $L_k''(0) = 0$. 

                \[L_k'''(x) = L_k''(x)(1-L_k(x)) + 2L_k'(x)(1-L_k'(x)) + L_k(x)(1-L_k''(x))\]

                Since $L_k''(0) = 0$ and $L_k'(0) = k/2$, we get:

                \[L_k'''(0) = 2L_k'(0)(1-L_k'(0)) + L_k(0) = \frac{1+2k(1-k)}{2}\]

                So the Taylor series of $L_k$ based at zero is:

                \[p(x) = L_k(0) + L_k'(0)x +L_k'(0)x^2/2 + L_k'''(0)x^3/3!\]
                \item 
                
                \item We're looking to maximize $|H(x)-L_k(x)|$ over the interval $[\epsilon, 1]$. Since $x > \epsilon >0$, $|H(x) - L_k(x)| = 1 - L_k(x)$. The derivative of $L_k$ is:

                \[L_k'(x) = 2k\frac{1}{1+e^{-2kx}}(1-\frac{1}{1+e^{-2kx}})\]

                Which is always positive. It follows that $1-L_k(x)$ is always decreasing, and so obtains a maximum at $x = \epsilon$. 
                So for $x \in [\epsilon, 1]$:

                \[|L_k(x)-H(x)| \leq 1 -\frac{1}{1+e^{-2k\epsilon}} = \frac{e^{-2k\epsilon}}{1 + e^{-2k \epsilon}}\]

                We can see that the limit will go to zero. 
                
                \end{enumerate}


                
			\end{enumerate}
			\item \begin{enumerate}
				\item The required Taylor polynomials are:
                \[e^x = 1 + x + \frac{1}{2}x^2 + \frac{1}{6}x^3\]
                \[\cos{x} = 1 - \frac{1}{2}x^2 + \frac{1}{24}x^4 \]
                \[\sin{x} = x - \frac{1}{6}x^3\]
                \[\cosh{x} = 1 + \frac{1}{2}x^2 + \frac{1}{4}x^4\]
                \[\sinh{x} = x + \frac{1}{6}x^3\]

                \item The required Taylor polynomials are:
                \[e^{ix} = 1 + ix - \frac{1}{2}x^2 - \frac{1}{6}x^3 + \frac{1}{24}x^4\]
                \[\cos{x} + i\sin{x} = 1 - \frac{1}{2}x^2 + \frac{1}{24}x^4 + ix - \frac{i}{6}x^3\]

                This demonstrates Euler's identity where $e^{ix} = \cos{x} + i\sin{x}$.

                \item The required Taylor polynomial is:
                \[\cosh{x} + \sinh{x} = 1 + \frac{1}{2}x^2 + \frac{1}{4}x^4 + x + \frac{1}{6}x^3\]

                Looking at the Taylor polynomial for $e^x$ from Part A, we can see that the identity $e^x = \cosh{x} + \sinh{x}$ holds.

			\end{enumerate}
		\end{enumerate}

	
	\end{solutions}
	\begin{instructions}
				\subsection*{Learning Objectives} Students need to be able to\ldots
		\begin{itemize}
			\item Create a spreadsheet that runs Euler's method to approximate the solution to a differential equation.
			\item Recognize the difference between the approximation suggested by Euler's method and the \emph{true} solution
			to a differential equation. This includes adjusting the step size of Euler's method to discern what is error vs. what
			is the underlying solution.
		\end{itemize}


		\subsection*{Context} In class we have created a spreadsheet to make an Euler approximation
		for an ODE. However, we haven't done this very many times. Students need more practice!

		This week we will start making Euler approximations for systems of ODEs using spreadsheets,
		so we need students to have a firm understanding of how to use spreadsheets and the errors
		that can arise.


		\subsection*{What to Do} This is the first tutorial of the term, and
		it is your chance to win the students over! This is a groupwork tutorial,
		but students may not be used to working in groups.

		\begin{itemize}
			\item Arranged for group work. Reorganize the desks and chairs
				(if possible) to facilitate groups of 3 or 4. Ask
				students to form groups of 3 or 4 with other students
				nearby. Don't allow larger groups.

			\item Begin the tutorial by introducing yourself (your name,
				your program of study, and your year). You might
				also want to give them some more personal information,
				such as where you are from or when you first started liking math.

			\item Introduce the structure and purpose of tutorials: students
				will be working to (1) better understand concepts
				from lecture, (2) practice tackling concepts that
				have not been explained in lecture, and (3) effectively
				communicate. They can expect to spend most of the
				tutorial working in small groups.

			\item Emphasize the importance of working with others when
				learning mathematics---they should be working with
				others in this tutorial \emph{and} outside of
				class.
		\end{itemize}

		This introduction should take no more than 5 minutes.

		Next, introduce the learning objectives for the day's tutorial. Explain
		that the goal of this tutorial. Their worksheet has the ``formal'' objectives
		stated and these instructions have the ``hidden'' objectives. Feel free
		to share with them the hidden objectives as well.

		Ask the students to pair up and
		start working on the problem list. Circulate around the room during
		this time and ask groups what they're thinking. They will be tempted
		to move quickly through the list without thoroughly checking their
		new answers---encourage them to think deeply.

		Problem 1 is a straightforward repetition of what we have done in class, but that doesn't mean
		students will know how to do it. Encourage them to outline how they would compute the first few steps
		of Euler's method before they make a spreadsheet. Then, encourage them to start a new spreadsheet from scratch rather
		than modify the spreadsheet we have from class. However, if a group is really struggling, you can give the opposite
		advice: look at a spreadsheet we used in class.

		There are too many problems to finish in 50 minutes and \emph{you should not be going
		over the solution to every problem}. Solutions will be posted for the students. The goal
		of tutorial is for students to spend time \emph{doing} mathematics with an expert around
		to help them if they get stuck. Don't feel any time pressure, even if you only get through 1.5
		questions, that's okay!

		During the last 6 minutes of class, pick one problem (perhaps a few parts of one problem)
		that most groups have at least started, and do this problem as a wrapup. Seeing an expert do the
		problem is the student's reward for working so hard.

		Since this tutorial involves making spreadsheets, a computer and projector setup is required.
		Check out your tutorial room before the first tutorial to see if it has the required technology.
		Also, rehearse making a spreadsheet for Question 1. Programming on the fly often doesn't end well.

		Notes:
		\begin{enumerate}
			\item For part (b), students may ignore the scale of the axes. They'll see a squiggle but fail to notice that the range of that
			squiggle is -0.001 to 0.001, which could be completely attributed to error.

			For part (c) they can manually scan through their data or try to guess from their graph. They can also use a spreadsheet formula
			like \verb|=IF(D5>MAX(D3:D4,D6:D7), 1, 0)| to indicate whether the current cell value is larger than the four surrounding cells.
			Then, they can scroll to find the position of the 1s.

			For part (d) they may not recognize that they can check concavity directly from the differential equation.

			\item For (b), have them actually give you the explanation. We're looking for something substantially more than ``the
			graphs get more and more periodic''. When we start doing systems of equations, this analysis will be important.

			\item Most groups will not have time to get to this problem. That's okay! The moral is that you won't always
			end up with increasing errors when using Euler's method (though often times you will), and only advanced students need
			to understand this subtlety.

			Make sure the class knows that its okay that they didn't get to this problem, but that it is good practice for them to do it at home.
		\end{enumerate}

	\end{instructions}

\end{document}
