
	
\begin{enumerate}
    \item \begin{enumerate}
        \item Though we cannot know exactly what the author of each answer was thinking, the steps do not appear to have logical contradictions.
        \item 
        \item Question (A1) has no explanation of what $A$ is (or $t$ for that matter). Neither (A2) nor (A3) have any words at all.
        They should be written in complete (mathematical) sentences.
        \item 

        \begin{enumerate}
            \item[(A1)] \emph{Write the complete solution to $y'=2y$. No work necessary.}

            \fbox{
            \begin{minipage}{0.78\textwidth}
                \begin{itemize}
                    \item[] $y=Ae^{2t}$ where $A\in\R$ is a parameter.
                \end{itemize}
            \end{minipage}
            }

            \item[(A2)] \emph{Let $f$ be a solution to $y'=2y$ satisfying $f(0)=1$. Use Euler's method to approximate $f(2)$. Explain how you arrived at your answer.}
            
            \fbox{
            \begin{minipage}{0.78\textwidth}
                \begin{itemize}
                    \item[] We will use a step size of $\Delta=0.5$.
                    \item[] Let $f_n$ be the approximation we get from the $n$th application of Euler's method. 
                    Since $f'=2f$, using a tangent line approximation, we get the recursive formula
                    \[
                        f_{n+1}=f_{n}+2f_{n}\Delta
                    \]
                    \item[] We start with initial condition $f(0)=1$. I.e., $f_0=1$.
                    \item[] After iterating four times, we get \[f(2)\approx 16.\]
                \end{itemize}
            \end{minipage}
            }

            \item[(A3)] \emph{Let $\vec r\,'(t)=M\vec r(t)$ be a differential equation and let $\vec p(t)$ and $\vec q(t)$
            be solutions. Show that $\vec s = \vec p+\vec q$ is also a solution.}

            
            \fbox{
            \begin{minipage}{0.78\textwidth}
            \begin{itemize}
                \item[] Because the derivative is linear, we have $\vec s\,'=\vec p\,'+\vec q\,'$.
                \item[] Note that $\vec p\,'=A\vec p$,
                \item[] and $\vec q\,'=A\vec q$.
                \item[] Thus, $\vec p\,'+\vec q\,' = A\vec p+A\vec q$.
                \item[] Because matrix multiplication is also linear, we have the identity $A(\vec p+\vec q) = A\vec p+A\vec q$.
                \item[] Combining this with our previous formula, we see $A\vec s=A(\vec p+\vec q)$,
                \item[] and so $\vec s\,'=A\vec s$, which means $\vec s$ is a solution to the differential equation.
            \end{itemize}
            \end{minipage}
            }
        \end{enumerate}
    \end{enumerate}

        \item \begin{enumerate}
            \item I would expect an opening statement explaining how bees can only pollinate flowers they can land on,
            an overview of the model stating that it was based on differential equations and simulated using Euler's method,
            and a summary of the model's results about how bees can only land on ``odd numbered'' flowers.

            \item \begin{enumerate}
                \item The AI wrote in complete sentences. It split the summary into sections with headings, and it 
                provided a summary of the results of the model.
                \item The AI did not explain how the model is relevant to the farmer's question about where to plant flowers.
                It used overly-technical equations, and is overall written in a boring style that does not encourage the reader to continue.
                \item No, the formulas included, while the basis for the model, are not self-explanatory. It takes simulations and phase 
                portraits to understand what is going on and so including the formulas in the executive summary is not helpful. Instead,
                the formulas belong in the body of the report. The executive summary itself should be more high-level, including things like
                ``a system of differential equations was used to model the bees; it was simulated using Euler's method\ldots''.

                \item \phantom{x}\begin{quote}
                    Bees are the primary pollinator of <your crop> and as common sense suggests, bees will only pollinate flowers
                    that they can land on. This report shows gives the details of a differential-equations based model predicting
                    that if you place a hive, flowers that grow at distances approximately $3.1$, $9.2$, $12.4$ metres away from the hive
                    will attract bees while other flowers will not. The findings suggest that you should follow specific planting patterns
                    to maximize pollination.
                \end{quote}
            \end{enumerate}
        \item
        \end{enumerate}
		

\end{enumerate}