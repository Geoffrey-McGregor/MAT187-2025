\begin{objectives}
	In this tutorial you explore the limits of what affine approximations can
	tell you about the nature of equilibrium solutions.

	These problems relate to the following course learning objectives:
	\textit{Apply linear algebra techniques to classify solutions of linear systems of ordinary differential
		equations including rigorously classifying the stability of equilibrium solutions and creating
		linear approximations to non-linear systems of ordinary differential equations}.
\end{objectives}

\subsection*{Problems}

Linearization is a powerful technique to classify critical points, but it can provide inconclusive results.

\begin{enumerate}
	\item Consider the differential equations
	      \begin{equation}
		      y'=-y^3-y \tag{A}
	      \end{equation}
	      \begin{equation}
		      y'=-y^3 \tag{B}
	      \end{equation}
	      \begin{equation}
		      y'=y^3 \tag{C}
	      \end{equation}
	      \begin{enumerate}
		      \item Use Desmos to graph the slope field for each equation. Based on the slope
		            field, how would you classify the equilibrium solution $y=0$ for each equation?

		            \url{https://www.desmos.com/calculator/ghavqzqqjn}
		            %\item Find the general solution to each differential equation and use the general solution to classify the equilibrium $y(t)=0$
		            %      as stable/unstable.
		      \item Find affine approximations for each differential equation centered at the equilibrium $y=0$.
		      \item Classify the critical points of each of your affine approximations as stable/unstable.
		      \item Why didn't your affine approximations correctly predict stability/instability in all cases? Explain.
		      \item In calculus, the \emph{first derivative test} states that at a point:
		            (i) if the derivative is positive, the function is increasing; (ii)
		            if				  the derivative is negative, the function is decreasing; and
		            (iii) if the derivative is zero, the test is inconclusive.

		            Make up your own ``affine approximation test'' for the stability/instability of an equilibrium solution
		            for a differential equation.
	      \end{enumerate}

	\item Let $\vec r(t)=\mat{x(t)\\y(t)}$ and consider the differential equation
	      \begin{equation}
		      \vec r\,' = \mat{-y-x^3\\x-y^3} \tag{D}
	      \end{equation}
	      which has $\vec r(t)=(0,0)$ as its only equilibrium solution.
	      \begin{enumerate}
		      \item Find an affine approximation Equation (D) centered at $(0,0)$.
		      \item Classify the nature of the equilibrium solution in your affine approximation.
		      \item Do you believe that the nature of the equilibrium solution in your affine approximation
		            is the same as the nature of the equilibrium solution in Equation (D)? Why or why not?
		      \item Make a phase portrait for Equation (D). Can you tell what the nature of the equilibrium solution is?

		            \textbf{Note:} Make sure to zoom in on your phase portrait before you draw a strong conclusion.
		      \item Use numerical simulations to classify the nature of the equilibrium solution.

		            \textbf{Hint:} You may need to use \emph{very} small step sizes to ensure that rounding error
		            doesn't obscure your results.

		            %\item Come up with an analytic argument to classify the nature of the equilibrium solution.

		      \item Suppose you have an affine approximation of a system of differential equations. Further suppose
		            that the matrix in your affine approximation has eigenvalues $a_1+b_1i$, $a_2+b_2i$, $a_3+b_3i$, \ldots, $a_n+b_ni$.
		            Explain how to classify an equilibrium solution as stable/unstable based on the eigenvalues.

	      \end{enumerate}
\end{enumerate}