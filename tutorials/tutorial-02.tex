\documentclass[red]{tutorial}
\usepackage[no-math]{fontspec}
\usepackage{xpatch}
	\renewcommand{\ttdefault}{ul9}
	\xpatchcmd{\ttfamily}{\selectfont}{\fontencoding{T1}\selectfont}{}{}
	\DeclareTextCommand{\nobreakspace}{T1}{\leavevmode\nobreak\ }
\usepackage{polyglossia} % English please
	\setdefaultlanguage[variant=us]{english}
%\usepackage[charter,cal=cmcal]{mathdesign} %different font
%\usepackage{avant}
\usepackage{microtype} % Less badboxes


\usepackage[charter,cal=cmcal]{mathdesign} %different font
%\usepackage{euler}
 
\usepackage{blindtext}
\usepackage{calc, ifthen, xparse, xspace}
\usepackage{makeidx}
\usepackage[hidelinks, urlcolor=blue]{hyperref}   % Internal hyperlinks
\usepackage{mathtools} % replaces amsmath
\usepackage{bbm} %lower case blackboard font
\usepackage{amsthm, bm}
\usepackage{thmtools} % be able to repeat a theorem
\usepackage{thm-restate}
\usepackage{graphicx}
\usepackage{xcolor}
\usepackage{multicol}
\usepackage{fnpct} % fancy footnote spacing
\usepackage{tikz}
\usetikzlibrary{arrows.meta}

\usepackage{pgfplots}
\pgfplotsset{compat=1.18}
%\pgfkeys{/pgf/fpu}

 
\newcommand{\xh}{{{\mathbf e}_1}}
\newcommand{\yh}{{{\mathbf e}_2}}
\newcommand{\zh}{{{\mathbf e}_3}}
\newcommand{\R}{\mathbb{R}}
\newcommand{\Z}{\mathbb{Z}}
\newcommand{\N}{\mathbb{N}}
\newcommand{\Proj}{\mathrm{proj}}
\newcommand{\Perp}{\mathrm{perp}}
\renewcommand{\span}{\mathrm{span}\,}
\newcommand{\Span}{\mathrm{span}\,}
\newcommand{\Img}{\mathrm{img}\,}
\newcommand{\Null}{\mathrm{null}\,}
\newcommand{\Range}{\mathrm{range}\,}
\newcommand{\rref}{\mathrm{rref}}
\newcommand{\Rank}{\mathrm{rank}}
\newcommand{\nnul}{\mathrm{nullity}}
\newcommand{\mat}[1]{\begin{bmatrix}#1\end{bmatrix}}
\renewcommand{\d}{\mathrm{d}}


\theoremstyle{definition}
\newtheorem{example}{Example}[section]
\newtheorem{defn}{Definition}[section]

%\theoremstyle{theorem}
\newtheorem{thm}{Theorem}[section]

\pgfkeys{/tutorial,
	name={Tutorial 2},
	author={},
	course={MAT 187},
	date={},
	term={},
	title={Sequences and Series}
	}

\begin{document}
	\begin{tutorial}
				
\begin{objectives}
	In this tutorial you practice using and manipulating sequences and series.
\end{objectives}

	\vspace{-1em}
\subsection*{Problems}
\begin{enumerate}
	\item % Some questions on summation notation
	% Some questions where tail sums are bounded
\end{enumerate}
	\end{tutorial}

	\begin{solutions}
				\begin{enumerate}
			\item
			\begin{enumerate}
				\item $(A(0),B(0)) = (0,1)$
				\item The solution is periodic because $(A(0),B(0))=(A(8),B(8))$
				\item \phantom{x}

				\begin{tikzpicture}
					\begin{axis}[
						title={$A$ vs. $B$},
						width=7cm,
						height=7cm,
						xmin=-2.5,xmax=5.5,
						ymin=-.5,
						ymax=5.5, xmajorgrids, ymajorgrids,
						xtick={-10,...,10}, ytick={0,1,...,10},
						axis lines=middle,
						samples=5, domain=-5:5,
						xlabel={$A$},
						ylabel={$B$}
						]
						
						%\addplot[red, thick] coordinates {(0,0) (2,4) (5,-1) (8,0)};
						%\addplot[green!50!black, thick] coordinates {(0,1) (2,4) (5,1) (8,1)};
						\addplot[magenta!50!black, very thick] coordinates {(0,1) (4,4) (-1,1) (0,1)};
					\end{axis}
				\end{tikzpicture}%
				
			\end{enumerate}
			\item \begin{enumerate}
				\item \phantom{x}

				\begin{tikzpicture}
					\begin{axis}[
						width=8cm,
						height=5cm,
						xmin=-1.8,xmax=6.8,			
						ymin=-3.5,
						ymax=3.5, xmajorgrids, ymajorgrids,
						xtick={-5,...,10}, ytick={-10,...,10},
						axis lines=middle,
						samples=5, domain=-5:5,
						xlabel={$t$},
						ylabel={$L$}
						]
						
						\addplot [
							domain=0:2*pi,
							samples=100,
							smooth,
							thick,
							blue
						] 
						({x}, {2*sign(cos(deg(x)))*abs(cos(deg(x)))^(1/2)});
						%({cos(deg(x)) / (1 + sin(deg(x))^2)}, {cos(deg(x)) * sin(deg(x)) / (1 + sin(deg(x))^2)});
					\end{axis}
				\end{tikzpicture}
				~~~~
				\begin{tikzpicture}
					\begin{axis}[
						width=8cm,
						height=7cm,
						xmin=-1.8,xmax=6.8,			
						ymin=-5.5,
						ymax=5.5, xmajorgrids, ymajorgrids,
						xtick={-5,...,10}, ytick={-10,...,10},
						axis lines=middle,
						samples=5, domain=-5:5,
						xlabel={$t$},
						ylabel={$K$}
						]
						
						\addplot [
							domain=0:2*pi,
							samples=100,
							smooth,
							thick,
							blue
						] 
						({x}, {5*sin(deg(x))});
						%({cos(deg(x)) / (1 + sin(deg(x))^2)}, {cos(deg(x)) * sin(deg(x)) / (1 + sin(deg(x))^2)});
					\end{axis}
				\end{tikzpicture}
				
				\item \phantom{x}

				\begin{tikzpicture}
					\begin{axis}[
						width=8cm,
						height=5cm,
						xmin=-1.8,xmax=6.8,			
						ymin=-3.5,
						ymax=3.5, xmajorgrids, ymajorgrids,
						xtick={-5,...,10}, ytick={-10,...,10},
						axis lines=middle,
						samples=5, domain=-5:5,
						xlabel={$t$},
						ylabel={$L$}
						]
						
						\addplot [
							domain=0:pi/2,
							samples=100,
							smooth,
							thick,
							blue
						] 
						({x}, {2*sign(cos(deg(2*x)))*abs(cos(deg(2*x)))^(1/2)});
						\addplot [
							domain=pi:2*pi,
							samples=100,
							smooth,
							thick,
							blue
						] 
						({x-pi/2}, {2*sign(cos(deg(x)))*abs(cos(deg(x)))^(1/2)});
						%({cos(deg(x)) / (1 + sin(deg(x))^2)}, {cos(deg(x)) * sin(deg(x)) / (1 + sin(deg(x))^2)});
					\end{axis}
				\end{tikzpicture}
				~~~~
				\begin{tikzpicture}
					\begin{axis}[
						width=8cm,
						height=7cm,
						xmin=-1.8,xmax=6.8,			
						ymin=-5.5,
						ymax=5.5, xmajorgrids, ymajorgrids,
						xtick={-5,...,10}, ytick={-10,...,10},
						axis lines=middle,
						samples=5, domain=-5:5,
						xlabel={$t$},
						ylabel={$K$}
						]
						
						\addplot [
							domain=0:pi/2,
							samples=100,
							smooth,
							thick,
							blue
						] 
						({x}, {5*sin(deg(2*x))});
						\addplot [
							domain=pi:2*pi,
							samples=100,
							smooth,
							thick,
							blue
						] 
						({x-pi/2}, {5*sin(deg(x))});
						%({cos(deg(x)) / (1 + sin(deg(x))^2)}, {cos(deg(x)) * sin(deg(x)) / (1 + sin(deg(x))^2)});
					\end{axis}
				\end{tikzpicture}
			\end{enumerate}

			\item \begin{enumerate}
				\item You can always create a graph in phase space from component graphs, but not the other way around.
				Graphs in phase space loose all ``speed'' information, but component graphs have information about the
				speed/velocity of a solution.

				\item Component graphs allow you to visualize the relationship among different quantities in ways
				that may not be obvious from analyzing component graphs. Component graphs, on the other hand, contain
				complete information about a solution, including its speed/velocity.
			\end{enumerate}

			\item
			\begin{enumerate}
				\item \phantom{x}

				\begin{center}
				\begin{tikzpicture}
					\begin{axis}[
						title={$t$ vs. $V(t)$},
						width=7cm,
						height=5cm,
						xmin=-5.5,xmax=5.5,
						ymin=-5.5,
						ymax=5.5, xmajorgrids, ymajorgrids,
						axis lines=middle,
						ytick={-5,...,5},
						samples=200, domain=-20:20, smooth]
						
						\addplot[red, thick] {5*sin(deg((pi-0.64)/(1+exp(-x))-3*pi/2+0.64))};
					\end{axis}
				\end{tikzpicture}%
				~~~~~~~\begin{tikzpicture}
					\begin{axis}[
						title={$t$ vs. $W(t)$},
						width=7cm,
						height=5cm,
						xmin=-5.5,xmax=5.5,
						ymin=-5.5,
						ymax=5.5, xmajorgrids, ymajorgrids,
						axis lines=middle,
						ytick={-5,...,5},
						samples=200, domain=-20:20, smooth]
						
						\addplot[green!50!black, thick] {-2*(abs(cos(deg((pi-0.64)/(1+exp(-x))-3*pi/2+0.64))))^(1/3)};
					\end{axis}
				\end{tikzpicture}
				\end{center}
				\item For any one-to-one and onto function $f$, the graph of $\vec r$ and $\vec r\circ f$ will be the same,
				with a possible change in domain. We want to change the domain from $(-3\pi/2+0.64, -\pi/2)$ to $\R$, which
				means we need to find a function that is one-to-one and onto from $\R$ to $(-3\pi/2+0.64, -\pi/2)$. 
				We can do this with the function
				\[
					d(t) = \frac{\pi-0.64}{1+e^{-x}}-\frac{3\pi}{2}+0.64.
				\]
				Then $\vec q=\vec r\circ d$.
				\item We can come up with a differential equation by computing $[\vec r\circ f]'$. Doing so we see
				\[
					V'(t) = \frac{12.508\, e^t \cos\left(4.07239 - \frac{2.50159 e^t}{1 + e^t}\right)}{(1 + e^t)^2}
				\]
				%(1.66773 E^x Sin[4.07239 - (2.50159 E^x)/(1 + E^x)])/((1. + E^x)^2 Cos[4.07239 - (2.50159 E^x)/(1 + E^x)]^(2/3))
				\[
				W'(t) = 
				\frac{1.66773\, e^t \sin\left(4.07239 - \frac{2.50159 e^t}{1 + e^t}\right)}{(1 + e^t)^2 \cos\left(4.07239 - \frac{2.50159 e^t}{1 + e^t}\right)^{\frac{2}{3}}}
				\]

				\item We can come up with a different set of equations by stretching the domain, for example, by multiplying by $2$.
				We then have
				\[
					V_{\text{new}}'(t) = [V(2t)]'=2V'(2t)
				\]
				\[
					W_{\text{new}}'(t) = [W(2t)]'=2W'(2t)
				\]
			\end{enumerate}
			
			%\item \begin{enumerate}
			%	\item The equilibrium points are labelled below.

			%	
			%	\hspace{-1.5em}\begin{tikzpicture}[scale=.6, blue]
			%		% Define the vector field function
			%		% based on
			%		% https://www.desmos.com/calculator/o3xjzddojw
			%		% x' = (x+2.87)*y*(x+8.63)
			%		% y' = x*(y+7.23)*(y-4.27)
			%		\def\vectorfieldx(#1,#2){(#1 + 2.87) * (#2) * (#1 + 8.63) / 200}
			%		\def\vectorfieldy(#1,#2){(#1) * (#2 + 7.23) * (#2 - 4.27) / 200}

			%		% Draw the vector field
			%		\foreach \x in {-11,-10.4,...,10}
			%			\foreach \y in {-10,-9.4,...,10}
			%			{
			%				% Calculate the vector at (\x, \y)
			%				\pgfmathsetmacro\vx{\vectorfieldx(\x,\y)}
			%				\pgfmathsetmacro\vy{\vectorfieldy(\x,\y)}
			%				
			%				% Normalize the vector for consistent arrow lengths
			%				\pgfmathsetmacro\norm{sqrt(\vx*\vx + \vy*\vy + 0.01)}
			%				\pgfmathsetmacro\scaleChange{atan(\norm)/100/\norm}
			%				\pgfmathsetmacro\vx{\vx*\scaleChange}
			%				\pgfmathsetmacro\vy{\vy*\scaleChange}
			%							
			%				% Draw the vector as an arrow
			%				\draw[->] (\x,\y) -- ++(\vx,\vy);
			%			}

			%			% add red dots at (0,0), (-2.87, -7.23), (-2.87, 4.27), (-8.63, -7.23), (-8.63, 4.27)
			%			\fill[red] (0,0) circle (0.2) node[above right,black, fill=white, xshift=0.2cm] {$A$};
			%			\fill[red] (-2.87, -7.23) circle (0.2) node[above right,black, fill=white, xshift=0.2cm] {$E$};
			%			\fill[red] (-2.87, 4.27) circle (0.2) node[above right,black, fill=white, xshift=0.2cm] {$B$};
			%			\fill[red] (-8.63, -7.23) circle (0.2) node[above right, black, fill=white, xshift=0.2cm] {$D$};
			%			\fill[red] (-8.63, 4.27) circle (0.2) node[above right, black, fill=white, xshift=0.2cm] {$C$};

			%	\end{tikzpicture}

			%	\item $A$ is \emph{stable}. $C$ is \emph{attracting} and \emph{stable}. $D$ is \emph{repelling} and \emph{unstable}.
			%	$E$ and $B$ are \emph{unstable}.
			%	
			%	\item There is only one attracting equilibrium solution, $C$. It's basin of attraction is labeled below.
			%	
			%	\hspace{-1.5em}\begin{tikzpicture}[scale=.5, blue]
			%		% Define the vector field function
			%		% based on
			%		% https://www.desmos.com/calculator/o3xjzddojw
			%		% x' = (x+2.87)*y*(x+8.63)
			%		% y' = x*(y+7.23)*(y-4.27)
			%		\def\vectorfieldx(#1,#2){(#1 + 2.87) * (#2) * (#1 + 8.63) / 200}
			%		\def\vectorfieldy(#1,#2){(#1) * (#2 + 7.23) * (#2 - 4.27) / 200}

			%		% Draw the vector field
			%		\foreach \x in {-11,-10.4,...,10}
			%			\foreach \y in {-10,-9.4,...,10}
			%			{
			%				% Calculate the vector at (\x, \y)
			%				\pgfmathsetmacro\vx{\vectorfieldx(\x,\y)}
			%				\pgfmathsetmacro\vy{\vectorfieldy(\x,\y)}
			%				
			%				% Normalize the vector for consistent arrow lengths
			%				\pgfmathsetmacro\norm{sqrt(\vx*\vx + \vy*\vy + 0.01)}
			%				\pgfmathsetmacro\scaleChange{atan(\norm)/100/\norm}
			%				\pgfmathsetmacro\vx{\vx*\scaleChange}
			%				\pgfmathsetmacro\vy{\vy*\scaleChange}
			%							
			%				% Draw the vector as an arrow
			%				\draw[->] (\x,\y) -- ++(\vx,\vy);
			%			}

			%			% add red dots at (0,0), (-2.87, -7.23), (-2.87, 4.27), (-8.63, -7.23), (-8.63, 4.27)
			%			\fill[red] (0,0) circle (0.2);
			%			\fill[red] (-2.87, -7.23) circle (0.2);
			%			\fill[red] (-2.87, 4.27) circle (0.2);
			%			\fill[red] (-8.63, -7.23) circle (0.2);
			%			\fill[red] (-8.63, 4.27) circle (0.2);

			%			\fill[green!50!black, opacity=0.5] (-11, 10) rectangle (-2.87, -7.23);
			%	\end{tikzpicture}
			%	
			%	\item Solutions in the bounded region have horizontal asymptotes as $x\to\infty$. Solutions in the left-half
			%	of the unbounded region below the bounded region have horizontal asymptotes as $x\to-\infty$. Solutions
			%	in the bounded region below the equilibrium points $B$ and $C$ have asymptotes as $x\to\pm\infty$.
			%	
			%	\hspace{-1.5em}\begin{tikzpicture}[scale=.5, blue]
			%		% Define the vector field function
			%		% based on
			%		% https://www.desmos.com/calculator/o3xjzddojw
			%		% x' = (x+2.87)*y*(x+8.63)
			%		% y' = x*(y+7.23)*(y-4.27)
			%		\def\vectorfieldx(#1,#2){(#1 + 2.87) * (#2) * (#1 + 8.63) / 200}
			%		\def\vectorfieldy(#1,#2){(#1) * (#2 + 7.23) * (#2 - 4.27) / 200}

			%		% Draw the vector field
			%		\foreach \x in {-11,-10.4,...,10}
			%			\foreach \y in {-10,-9.4,...,10}
			%			{
			%				% Calculate the vector at (\x, \y)
			%				\pgfmathsetmacro\vx{\vectorfieldx(\x,\y)}
			%				\pgfmathsetmacro\vy{\vectorfieldy(\x,\y)}
			%				
			%				% Normalize the vector for consistent arrow lengths
			%				\pgfmathsetmacro\norm{sqrt(\vx*\vx + \vy*\vy + 0.01)}
			%				\pgfmathsetmacro\scaleChange{atan(\norm)/100/\norm}
			%				\pgfmathsetmacro\vx{\vx*\scaleChange}
			%				\pgfmathsetmacro\vy{\vy*\scaleChange}
			%							
			%				% Draw the vector as an arrow
			%				\draw[->] (\x,\y) -- ++(\vx,\vy);
			%			}

			%			% add red dots at (0,0), (-2.87, -7.23), (-2.87, 4.27), (-8.63, -7.23), (-8.63, 4.27)
			%			\fill[red] (0,0) circle (0.2);
			%			\fill[red] (-2.87, -7.23) circle (0.2);
			%			\fill[red] (-2.87, 4.27) circle (0.2);
			%			\fill[red] (-8.63, -7.23) circle (0.2);
			%			\fill[red] (-8.63, 4.27) circle (0.2);

			%			% add label "bounded" at the center of the rectangle

			%			\fill[green!50!black, opacity=0.5] (-11, 10) rectangle (-2.87, -7.23);
			%			\node[above, black, fill=white, xshift=0.2cm] at (-6.935, 1.385) {bounded};
			%			
			%			\fill[orange!50!black, opacity=0.5] (-2.87, -7.23) rectangle (10, 4.27);
			%			\node[above, black, fill=white, xshift=0.2cm] at (3.065, 0) {bounded and periodic};

			%			\fill[red!50!black, opacity=0.5] (-11, -7.23) rectangle (10, -10);
			%			\node[above, black, fill=white, xshift=0.2cm] at (0, -10) {unbounded};
			%			
			%			\fill[red!50!black, opacity=0.5] (-2.87, 4.27) rectangle (10, 10);
			%			\node[above right, black, fill=white, xshift=0.2cm] at (0, 4.5) {unbounded};
			%	\end{tikzpicture}
			%\end{enumerate}
		\end{enumerate}
	
	\end{solutions}
	\begin{instructions}
		\subsection*{Learning Objectives}
	Students need to be able to\ldots
	\begin{itemize}
		\item Switch between phase portraits and component graphs.
		\item Recognize the difference between solutions and curves in phase space.
		\item Deduce properties of equilibrium solutions from phase portraits.
	\end{itemize}

\subsection*{Context}
	We have analyzed systems of ODEs in class and in tutorial using phase portraits. We are getting ready to
	study systems of ODEs and equilibrium solutions to system written in matrix form. This requires a very good
	understanding of phase portraits and how they relate to solutions to systems.

\subsection*{What to Do}
	Introduce the learning objectives for the day's tutorial. Explain that phase portraits/graphs in phase space are an important
	tool in the study of differential equations and we want to better understand how graphs in phase space
	relate to solutions to systems of ODEs.

	Start by asking students to recall what we mean when we say ``phase portrait'' and ``component space'' for a system of ODEs
	$A'(t)=\ldots$ and $B'(t)=\ldots$. \emph{Do not give a lecture on this}, but you may have a short discussion ($< 5$ min)
	on this distinction.
	
	After most groups have finished \#1, go over it as a class. Doing so will ensure everyone has a baseline understanding
	of the distinction between phase space and component space.

	Continue as usual, walking around the room and asking
		questions while letting students work on the next problem and gathering them together
		for discussion when most groups have finished.

		7 minutes before class ends, pick a suitable problem to do as a wrap-up. Most likely, \#3 will be a good choice.



	
\subsection*{Notes}
	\begin{enumerate}
		\item Hopefully this problem won't be hard for them. All the vertices line up, after all. If they are struggling,
		it is worth spending a lot of time on this question, since they cannot use phase portraits unless they understand this.

		\item Students will struggle getting a curve for this question. If they are struggling, have them approximate
		the curve with a polygon and try to find component graphs for the polygon first. Then they can ``smooth it out''
		to get the curve.
		
		Part (b) will be especially hard. Some will struggle to even understand what the question is asking.

		\item This question is a good wrap-up question. It should be quicker than all the other questions.

		\item This is a question for groups who have moved more quickly. It will be very hard for the students and requires a
		mastery of pre-calculus concepts. Students may need some prodding on part (b) about stretching the domain and what
		types of functions will give them an appropriate stretch/compression.

		%\item This is a question for groups who have moved more quickly. Students may have forgotten what an equilibrium
		%solution is. Ask them to check their notes for the definition.

		%Part (c) introduces a brand new definition. Assure students that they have not learned this definition in class
		%and that's okay!

	\end{enumerate}
	\end{instructions}

\end{document}
