		\begin{objectives}
			In this tutorial, you practice creating and using polynomial approximations of
			functions.
		\end{objectives}

	%	\bigskip


		\subsection*{Problems}
		
		\begin{enumerate}
			\item  The inverse square root function, $f(x) = 1/\sqrt{x}$ is an important function
			when simulating light and reflections in computer graphics.  However, computing square 
			roots can be expensive, so we will use much-simpler polynomial approximations.
				\begin{enumerate}
					\item Find the first and second degree Taylor approximations to $f$ at $x=4$.
					Use each of these polynomials to approximate $f(2)$. Use a calculator 
					to quantify the error in each approximation.
					\item Find the quadratic polynomial that interpolates between the points 
					$(3,f(3))$, $(4,f(4))$, and $(5,f(5))$. 
					Use this polynomial to approximate $f(2)$ and quantify its error.

					\item Use Desmos to graphically answer the following: 
					In what region(s) is the interpolating polynomial better?
					In what region(s) is the Taylor polynomial better?

					\item Find the third and fourth degree Taylor approximations to $f$ at $x=4$.
					How well do they approximate $f(0.5)$. What about $f(7.5)$?

					\item % Finish off with insights about the asymptote blocking convergence
				\end{enumerate}

				\item % Try to approximate a step function? Let them use either
				% interpolation or the Taylor series of something that becomes just like the 
				% step function? This could be fund and frustrating!
                Try to approximate a step function.

				\item % A question where they need to approximate a function on R^2 by
				% using a complex polynomial and then taking the real and imaginary parts
				% would be neat.
		\end{enumerate}
