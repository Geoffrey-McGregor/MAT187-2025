\begin{enumerate}
		\item \begin{enumerate}
			\item We know that $\vec 0\in \Null(M)$. We also know that $\vec u-\vec v=\mat{3\\-1}\in \Null(M)$. Since $\Null(M)$
			is a subspace, we also have that $2\mat{3\\-1}\in\Null(M)$, along with many others.
			\item We can add a vector in the null space to any existing solution. For example, $\vec u+\mat{3\\-1}$, $\vec u+\mat{6\\-2}$, 
			and $\vec u+\mat{9\\-3}$ are all solutions.
			\item No. Unless $M\vec 0=\vec b$, we do not expect the solution set to the a subspace. If the solution set is not a subspace
			it is unlikely that summing two random solutions will result in another solution.
		\end{enumerate}
		
		\item \begin{enumerate}
			\item Differentiating, we see $u''+u=t^2-2$ and $v''+v=t^2-2$.
			\item By computing $u(t)-v(t)=-2\sin t$, we can guess that $u-2\sin$, $u-4\sin$, and $u-6\sin$ are all solutions. Computing,
			we verify that they indeed are.
			\item The transformation can be described in words by ``Take the second derivative of the function and then add one copy of the original function''.
			\item Similarly to the previous problem, we do not expect the solution set to be a subspace. However, with the transformation described in
			part (c), we can realize $t^2-2+K\sin t$ as a particular solution, $t^2-2$, plus a multiple of something in the null space, $\sin t$.
		\end{enumerate}
		
		\item \begin{enumerate}
			\item Let $f,g\in \mathcal C$ and let $k$ be a scalar. Then 
			\[\Id(f+g)=f+g=\Id(f)+\Id(g)\]
			similarly,
			\[
				\Id(kf)=kf=k\Id(f)
			\]
				and so $\Id$ is linear.

			For $D^2$, we compute
			\[
				D^2(f+g)=(f+g)''=f''+g''=D^2(f)+D^2(g)
			\]
			 and 
			 \[
			 	D^2(kf)=(kf)''=k(f'')=kD^2(f)
			 \]
				and so $D^2$ is linear.

			\item The sum of linear transformation is linear, so $T$ is linear.
			\item Computing $T(\sin) = -\sin+\sin=0$ and $T(\cos) = -\cos+\cos=0$ and so $\sin,\cos\in \Null(T)$.
			\item First note that $\{\sin,\cos\}$ is a linearly independent set. Therefore $\Null(T)=\Span\{\sin,\cos\}$.

			Now, notice that $t^2-2$ is a particular solution to $T(f)=t^2$. Because the equation $T(f)=t^2$ is a linear equation,
			we can write the complete solution as $\text{particular}+\Null(T)$ and so the complete solution to $T(f)=t^2$ is
			\[
				\{t^2-2\}+\Span\{\sin t,\cos t\}.
			\]
			Finally, notice that solutions to $T(f)=t^2$ are the same thing as solutions to $y''+y=t^2$.
		\end{enumerate}
\end{enumerate}
	
