\documentclass[red]{tutorial}
\usepackage[no-math]{fontspec}
\usepackage{xpatch}
	\renewcommand{\ttdefault}{ul9}
	\xpatchcmd{\ttfamily}{\selectfont}{\fontencoding{T1}\selectfont}{}{}
	\DeclareTextCommand{\nobreakspace}{T1}{\leavevmode\nobreak\ }
\usepackage{polyglossia} % English please
	\setdefaultlanguage[variant=us]{english}
%\usepackage[charter,cal=cmcal]{mathdesign} %different font
%\usepackage{avant}
\usepackage{microtype} % Less badboxes

%\usepackage{enumitem}

\usepackage[charter,cal=cmcal]{mathdesign} %different font
%\usepackage{euler}
 
\usepackage{blindtext}
\usepackage{calc, ifthen, xparse, xspace}
\usepackage{makeidx}
\usepackage[hidelinks, urlcolor=blue]{hyperref}   % Internal hyperlinks
\usepackage{mathtools} % replaces amsmath
\usepackage{bbm} %lower case blackboard font
\usepackage{amsthm, bm}
\usepackage{thmtools} % be able to repeat a theorem
\usepackage{thm-restate}
\usepackage{graphicx}
\usepackage[dvipsnames]{xcolor}
\usepackage{multicol}
\usepackage{fnpct} % fancy footnote spacing
\usepackage{tikz}
\usetikzlibrary{arrows.meta}

\usepackage{pgfplots}
\pgfplotsset{compat=1.18}
%\pgfkeys{/pgf/fpu}

 
\newcommand{\xh}{{{\mathbf e}_1}}
\newcommand{\yh}{{{\mathbf e}_2}}
\newcommand{\zh}{{{\mathbf e}_3}}
\newcommand{\R}{\mathbb{R}}
\newcommand{\Z}{\mathbb{Z}}
\newcommand{\N}{\mathbb{N}}
\newcommand{\proj}{\mathrm{proj}}
\newcommand{\Proj}{\mathrm{proj}}
\newcommand{\Perp}{\mathrm{perp}}
\renewcommand{\span}{\mathrm{span}\,}
\newcommand{\Span}{\mathrm{span}\,}
\newcommand{\Img}{\mathrm{img}\,}
\newcommand{\Null}{\mathrm{null}\,}
\newcommand{\Range}{\mathrm{range}\,}
\newcommand{\rref}{\mathrm{rref}}
\newcommand{\rank}{\mathrm{rank}}
\newcommand{\Rank}{\mathrm{rank}}
\newcommand{\nnul}{\mathrm{nullity}}
\newcommand{\mat}[1]{\begin{bmatrix}#1\end{bmatrix}}
\newcommand{\chr}{\mathrm{char}}
\renewcommand{\d}{\mathrm{d}}


\theoremstyle{definition}
\newtheorem{example}{Example}[section]
\newtheorem{defn}{Definition}[section]

%\theoremstyle{theorem}
\newtheorem{thm}{Theorem}[section]

\pgfkeys{/tutorial,
	name={Tutorial 6},
	author={},
	course={MAT 187},
	date={},
	term={},
	title={Parametric Equations}
	}

\begin{document}
	\begin{tutorial}
				\begin{objectives}
	In this tutorial you will be exploring the relationship between real and complex solutions to differential equations.

	These problems relate to the following course learning objectives:
			\textit{
				Apply linear algebra techniques to classify solutions of linear systems of ordinary differential
equations including rigorously classifying the stability of equilibrium solutions and creating
linear approximations to non-linear systems of ordinary differential equations}.
		\end{objectives}


\subsection*{Problems}

For this tutorial, you will be using Euler's formula (Not to be confused with Euler's method)
$
	e^{i\theta}=\cos \theta + i\sin \theta
$ and the fact that the complex conjugate of a number $a+bi$ is $a-bi$.


\begin{enumerate}
	\item Consider the differential equation $y''=-y$.
	\begin{enumerate}
		\item Show that $f(t)=e^{it}$ is a solution to $y''=-y$.
		\item Should the equation $y''=-y$ have a \emph{real} solution? Why or why not?
		\item\label{sin} Use Euler's formula to find the real part of $f$. Is the real part of $f$ still a solution? Justify.
		\item\label{cos} Use Euler's formula to find the imaginary part of $f$. Is the imaginary part of $f$ still a solution? Justify.
		\item The solution space to $y''=-y$ is two dimensional and has a basis $\{e^{it}, e^{-it}\}$. Express your
		solutions to parts \ref{sin} and \ref{cos} in terms of this basis.
		\item\label{realsoln} If a \emph{real polynomial} (i.e., a polynomial with only real coefficients) has a complex solution $a+bi$, then
		the complex conjugate $a-bi$ is also a solution. Suppose that a real differential equation has solutions $e^{(a+bi)t}$
		and $e^{(a-bi)t}$. Prove that this differential equation also has two linearly independent real solutions.
		\item Consider the equation $y'=iy$. Find a solution, $g$, to this differential equation. Is the real part of $g$ also a solution? What 
		about the imaginary part? Explain.
	\end{enumerate}

	\item Let $M=\mat{0&-1\\1&0}$ and consider the differential equation $\vec r\,'=M\vec r$.
		\begin{enumerate}
			\item Make a phase portrait for this system. Based on the phase portrait, do you think this solution has real solutions?
			\item Find the eigenvectors and eigenvalues of $M$.
			\item The solution space to $\vec r\,'=M\vec r$ is two dimensional. Find a basis of eigen solutions for $\vec r\,'=M\vec r$.
			\item Find a basis of real solutions for $\vec r\,'=M\vec r$.
			\item Is there a real eigen solution to $\vec r\,'=M\vec r$? Why or why not?
		\end{enumerate}
\end{enumerate}
	\end{tutorial}

	\begin{solutions}
		\begin{enumerate}
		\item \begin{enumerate}
			\item Differentiating $f'=ie^{it}$ and so $f''=i^2e^{it}=-e^{it}=-f$.
			\item Yes. The equation is a real equation and so it should have a real solution.
			\item By Euler's formula, $e^{it}=\cos t + i\sin t$, so the real part of $f$ is $\cos t$. 

			Differentiating, we see $\cos '' = -\cos$, so it is still a solution.
			\item By Euler's formula, $e^{it}=\cos t+i\sin t$, so the imaginary part of $f$ is $\sin t$.

			Differentiating, we see $\sin '' = -\sin$, so it is still a solution.
			\item Again, we can apply Euler's formula. We see
			\[
				e^{it}=\cos t+i\sin t \qquad\text{and}\qquad e^{-it}=\cos(-t) + i\sin(-t)=\cos t-i\sin t,
			\]
			so
			\[
				\tfrac{1}{2}e^{it}+\tfrac{1}{2}e^{-it}=\cos t.
			\]
			Similarly,
			\[
				\tfrac{-i}{2}e^{it}+\tfrac{i}{2}e^{-it}=\sin t.
			\]
			
			\item Again, we can apply Euler's formula.
			\[
				e^{(a+bi)t}=e^{at}e^{ibt}=e^{at}(\cos (bt)+i\sin (bt))
				\qquad\text{and}\qquad
				e^{(a-bi)t}=e^{at}e^{-ibt}=e^{at}(\cos (bt)-i\sin (bt))
			\]
			and so
			\[
				\tfrac{1}{2}e^{(a+bi)t}+\tfrac{1}{2}e^{(a-bi)t}=e^{at}\cos(bt)
			\]
				and 
			\[
				\tfrac{-i}{2}e^{(a+bi)t}+\tfrac{i}{2}e^{(a-bi)t}=e^{at}\sin(bt).
			\]
			Notice that $e^{at}\cos(bt)$ and $e^{at}\sin(bt)$ are linearly independent functions that are real.

			\item A solution is $g(t)=e^{it}$. The real part of $g(t)$ is $\cos t$ and the imaginary part is $\sin t$.
			Neither of these are solutions to $y'=iy$. Since $y'=iy$ is not a real equation, we don't expect real solutions,
			so we wouldn't expect the real and imaginary parts of $g$ to be a solution.

		\end{enumerate}

		\item
		\begin{enumerate}
			\item The phase portrait shows arrows going in a circle. There should be real solutions to this equation,
			because Euler's method will simulate a solution and there is no reason to expect that Euler's method will
			not converge.
			\item The eigenvalues of $M$ are $\pm i$ and the eigenvectors are $\mat{1\\-i}$ with eigenvalue $i$ and $\mat{1\\i}$
			with eigenvalue $-i$.
			\item The eigen solutions are $e^{it}\mat{1\\-i}$ and $e^{-it}\mat{1\\i}$. They are linearly independent and so form a basis.
			\item We can find a real basis either by finding a linear combination of the complex eigen solutions that is real or by 
			guessing and checking. We can guess that the real and imaginary parts of the complex eigen solutions are real solutions.
			
			Expanding with Euler's formula, we see
			\[
				e^{it}\mat{1\\-i}=\mat{e^{it}\\-ie^{it}}=\mat{\cos t+i\sin t\\\sin t-i\cos t},
			\]
			and so we guess that $\vec s_1(t) = \mat{\cos t\\\sin t}$ and $\vec s_2(t)=\mat{\sin t\\-\cos t}$ are solutions. Differentiating,
			we indeed see that these are solutions. Since they are real, linearly independent, and we have two of them, we have found a basis
			of real solutions.
			\item There is no real eigen solution to $\vec r\,'=M\vec r$. We have already found a basis of eigen solutions to the equation.
			All other eigen solutions must be multiples of the eigen solutions we found. However, there is no complex eigen solution that will
			make $e^{it}\mat{1\\-i}$ or $e^{it}\mat{1\\i}$ real.
		\end{enumerate}

\end{enumerate}
	
	\end{solutions}
	\begin{instructions}
		\subsection*{Learning Objectives}
	Students need to be able to\ldots
	\begin{itemize}
		\item Use Euler's formula to find the real and imaginary parts of complex solutions.
		\item Use complex solutions to find real solutions to real differential equations.
	\end{itemize}

\vspace*{-.5cm}
\subsection*{Context}

In class we have seen matrix differential equations with complex eigen solutions. We have found real eigen solutions
based on these, but we haven't done the ``hard part'' of exploring why exactly taking the real/imaginary parts of complex
solutions provides us with real solutions (of course this doesn't always work, but it often does).

This tutorial is chance for students to work through the details of these complex-number computations.

\subsection*{What to Do}
	Introduce the learning objectives for the day's tutorial. Explain that using complex numbers is a useful 
	intermediate step for solving differential equations, even if the solution in the end should be real. Remind the
	students what Euler's formula is (it's written at the top of their worksheet) and briefly go over what the real/imaginary 
	part of a complex number is. Stress that the \emph{imaginary} part of a complex number is actually a \emph{real} number (i.e., you
	drop the $i$).
	
	Have students get into groups and start on \#1. This question has many parts and may take most students the whole tutorial.
	Some students may confuse Euler's formula with Euler's method. Let these students know that these are different things, and Euler's
	formula is provided at the top of the worksheet.
	
	1e and 1g are good questions for wrapping up.

\subsection*{Notes}
	\begin{enumerate}
		\item Students may think that the imaginary part of $a+bi$ is $bi$. Correct them and let them know that it is just $b$.
		\item This question looks a lot like what we do in class. It should be straight forward after they did \#1, but it may be time consuming.
	\end{enumerate}
	\end{instructions}

\end{document}
