\documentclass[red]{tutorial}
\usepackage[no-math]{fontspec}
\usepackage{xpatch}
	\renewcommand{\ttdefault}{ul9}
	\xpatchcmd{\ttfamily}{\selectfont}{\fontencoding{T1}\selectfont}{}{}
	\DeclareTextCommand{\nobreakspace}{T1}{\leavevmode\nobreak\ }
\usepackage{polyglossia} % English please
	\setdefaultlanguage[variant=us]{english}
%\usepackage[charter,cal=cmcal]{mathdesign} %different font
%\usepackage{avant}
\usepackage{microtype} % Less badboxes


\usepackage[charter,cal=cmcal]{mathdesign} %different font
%\usepackage{euler}
 
\usepackage{tikz}
\usepackage{pgfplots}
\usetikzlibrary{arrows.meta}

\usepackage{blindtext}
\usepackage{calc, ifthen, xparse, xspace}
\usepackage{makeidx}
\usepackage[hidelinks, urlcolor=blue]{hyperref}   % Internal hyperlinks
\usepackage{mathtools} % replaces amsmath
\usepackage{bbm} %lower case blackboard font
\usepackage{amsthm, bm}
\usepackage{thmtools} % be able to repeat a theorem
\usepackage{thm-restate}
\usepackage{graphicx}
\usepackage{xcolor}
\usepackage{multicol}
\usepackage{fnpct} % fancy footnote spacing

 
\newcommand{\xh}{{{\mathbf e}_1}}
\newcommand{\yh}{{{\mathbf e}_2}}
\newcommand{\zh}{{{\mathbf e}_3}}
\newcommand{\R}{\mathbb{R}}
\newcommand{\Z}{\mathbb{Z}}
\newcommand{\N}{\mathbb{N}}
\newcommand{\proj}{\mathrm{proj}}
\newcommand{\Proj}{\mathrm{proj}}
\newcommand{\Perp}{\mathrm{perp}}
\newcommand{\Span}{\mathrm{span}\,}
\newcommand{\Img}{\mathrm{img}\,}
\newcommand{\Null}{\mathrm{null}\,}
\newcommand{\Range}{\mathrm{range}\,}
\newcommand{\rref}{\mathrm{rref}}
\newcommand{\Rank}{\mathrm{rank}}
\newcommand{\nnul}{\mathrm{nullity}}
\newcommand{\mat}[1]{\begin{bmatrix}#1\end{bmatrix}}
\renewcommand{\d}{\mathrm{d}}
\newcommand{\Id}{\operatorname{id}}


\theoremstyle{definition}
\newtheorem{example}{Example}[section]
\newtheorem{defn}{Definition}[section]

%\theoremstyle{theorem}
\newtheorem{thm}{Theorem}[section]

\pgfkeys{/tutorial,
	name={Tutorial 4},
	author={},
	course={MAT 187},
	date={},
	term={},
	title={Integration Techniques}
	}

\begin{document}
	\begin{tutorial}
				\begin{objectives}
			In this tutorial you will explore some of the ways that Linear Algebra techniques can be applied to the
			study of differential equations.

				These problems relate to the following course learning objectives:
						\textit{Apply linear algebra techniques to classify solutions of linear systems of ordinary differential
			equations including rigorously classifying the stability of equilibrium solutions and creating
			linear approximations to non-linear systems of ordinary differential equations.}
		\end{objectives}

		\vspace{-.5em}
		\subsection*{Problems}
		\vspace{-.5em}

	%	In MAT223, you studied linear algebra in the context of $\R^n$ where vectors were geometric objects. We can expand our notion of 
	%	vectors to include \emph{functions} from $\R$ to $\R$ (you can think of functions as vectors in $\R^{\infty}$).

\begin{enumerate}
	\item\label{q1}
	Recall from Linear Algebra that for a matrix $M$, the complete solution to the equation $M\vec x=\vec b$ can be expressed as
	$
		\Null(M)+\{\vec p\}
	$
		where $\vec p$ is a particular solution to $M\vec x=\vec b$.
		
	\begin{enumerate}
		\item Suppose $\vec u=\mat{1\\2}$ and $\vec v=\mat{-2\\3}$ are two solutions to $M\vec x=\vec b$. Use this information
		to find at least three vectors in $\Null(M)$.
		\item Based on the information above, find three more solutions to $M\vec x=\vec b$.
		\item Do you expect $\vec u+\vec v$ to be a solution to $M\vec x=\vec b$? Why or why not?
	\end{enumerate}

	\item Consider the differential equation
	\[
		y''+y=t^2
	\]
	\begin{enumerate}
		\item Show that $u(t)=t^2-2+\sin t$ and $v(t)=t^2-2+3\sin t$ are solutions to the differential equation.
		\item Use a process similar to what you did in Question \ref{q1} to guess three additional solutions to the differential equation. Verify
		whether or not your guesses are actually solutions.
		\item The function $u(t)-v(t)=-2\sin t$ can be considered to be in the ``null space'' of some transformation. What is this transformation?
		\item Let $K\neq 1$. Why is $t^2-2+K\sin t$ a solution to the differential equation but $K(t^2-2+\sin t)$ not? Explain using linear algebra concepts.
	\end{enumerate}

	\item In this question, we will continue exploring the differential equation $y''+y=t^2$.
	
	Let $\mathcal C=\{\text{infinitely differentiable functions from $\R$ to $\R$}\}$, let $\Id:\mathcal C\to\mathcal C$
	be the identity transformation, and let $D^2:\mathcal C\to\mathcal C$ be the transformation that sends a function to its second derivative.
	Additionally,define $T=D^2+\Id$.

	\begin{enumerate}
		\item Show that $\Id$ and $D^2$ are linear transformations. (Use the same definition of Linear Transformation as in your Linear Algebra courses.)
		\item Show that $T$ is a linear transformation.
		\item Find $\Null(\Id)$ and $\Null(D^2)$.
		\item Show that $\sin$ and $\cos$ are in $\Null(T)$.
		\item You may take as a fact that $\Null(T)$ is two dimensional. Using this fact, find the complete solution to $y''+y=t^2$. Justify your steps.
	\end{enumerate}

\end{enumerate}


















	\end{tutorial}

	\begin{solutions}
		\begin{enumerate}
		\item \begin{enumerate}
			\item We know that $\vec 0\in \Null(M)$. We also know that $\vec u-\vec v=\mat{3\\-1}\in \Null(M)$. Since $\Null(M)$
			is a subspace, we also have that $2\mat{3\\-1}\in\Null(M)$, along with many others.
			\item We can add a vector in the null space to any existing solution. For example, $\vec u+\mat{3\\-1}$, $\vec u+\mat{6\\-2}$, 
			and $\vec u+\mat{9\\-3}$ are all solutions.
			\item No. Unless $M\vec 0=\vec b$, we do not expect the solution set to the a subspace. If the solution set is not a subspace
			it is unlikely that summing two random solutions will result in another solution.
		\end{enumerate}
		
		\item \begin{enumerate}
			\item Differentiating, we see $u''+u=t^2-2$ and $v''+v=t^2-2$.
			\item By computing $u(t)-v(t)=-2\sin t$, we can guess that $u-2\sin$, $u-4\sin$, and $u-6\sin$ are all solutions. Computing,
			we verify that they indeed are.
			\item The transformation can be described in words by ``Take the second derivative of the function and then add one copy of the original function''.
			\item Similarly to the previous problem, we do not expect the solution set to be a subspace. However, with the transformation described in
			part (c), we can realize $t^2-2+K\sin t$ as a particular solution, $t^2-2$, plus a multiple of something in the null space, $\sin t$.
		\end{enumerate}
		
		\item \begin{enumerate}
			\item Let $f,g\in \mathcal C$ and let $k$ be a scalar. Then 
			\[\Id(f+g)=f+g=\Id(f)+\Id(g)\]
			similarly,
			\[
				\Id(kf)=kf=k\Id(f)
			\]
				and so $\Id$ is linear.

			For $D^2$, we compute
			\[
				D^2(f+g)=(f+g)''=f''+g''=D^2(f)+D^2(g)
			\]
			 and 
			 \[
			 	D^2(kf)=(kf)''=k(f'')=kD^2(f)
			 \]
				and so $D^2$ is linear.

			\item The sum of linear transformation is linear, so $T$ is linear.
			\item Computing $T(\sin) = -\sin+\sin=0$ and $T(\cos) = -\cos+\cos=0$ and so $\sin,\cos\in \Null(T)$.
			\item First note that $\{\sin,\cos\}$ is a linearly independent set. Therefore $\Null(T)=\Span\{\sin,\cos\}$.

			Now, notice that $t^2-2$ is a particular solution to $T(f)=t^2$. Because the equation $T(f)=t^2$ is a linear equation,
			we can write the complete solution as $\text{particular}+\Null(T)$ and so the complete solution to $T(f)=t^2$ is
			\[
				\{t^2-2\}+\Span\{\sin t,\cos t\}.
			\]
			Finally, notice that solutions to $T(f)=t^2$ are the same thing as solutions to $y''+y=t^2$.
		\end{enumerate}
\end{enumerate}
	

	
	\end{solutions}
	\begin{instructions}
		\subsection*{Learning Objectives}
	Students need to be able to\ldots
	\begin{itemize}
		\item Recall and use Linear Algebra concepts from MAT223
		\item Use existing solutions to find other solutions to linear ODEs
	\end{itemize}

\subsection*{Context}
	We have just started applying linear algebra concepts to MAT244. We have written systems of first-order ODEs
	in matrix form and are exploring how eigenvectors of that matrix relate to eigen solutions and general solutions.
	However, students are very fuzzy on their Linear Algebra. About 1/3rd of the students are currently taking MAT224,
	but many of the others haven't seen Linear Algebra in at least 6 months. And, their foundation may have been shaky to begin with.

\subsection*{What to Do}
	Introduce the learning objectives for the day's tutorial. Explain that linear algebra concepts are
	very useful in analyzing and solving differential equations. Also explain that while you can look up
	any concepts you've forgotten, it is much better for memory to struggle to remember on your own \emph{before}
	you look up concepts you may have forgotten.

	Have students pair up and start with a \textbf{Warmup:}
	\begin{itemize}
		\item Write down the definition of the \emph{null space} of a linear transformation.
	\end{itemize}

		Many will struggle. After 3 minutes, discuss the definition as a whole class (it is necessary to make progress on \#1).

		After the warmup, have the students work in groups on the remaining problems. Circulate and ask/answer questions as usual.

		7 minutes before the end of class, pick a problem that most students have started working on
		to do as a wrap-up.

\subsection*{Notes}
	\begin{enumerate}
			\item In part (a) students may need a hint. Ask them: ``What does $\vec u$ correspond to in the expression $\Null(M)+\{\vec p\}$?''
			Part (c) we don't have enough information to answer definitively (the vector $\vec b$ could be the zero vector, after all), but we can
			answer in the ``general'' case.

			\item Part (a) should be straight forward. Part (b) will make many students confused and uncomfortable. We're asking them to take
			a leap of faith. The wording of the problem is to ``guess'' a solution, not to ``find'' a solution. Remind them that it is very quick
			to check if a guess is correct, so they can try lots of things.

			Part (d) can be answered by actually differentiating the expressions or with a hand-wavy answer like 1(c).

			\item This question steps the abstraction up and is much harder. It is there fore the students who have a solid linear
			algebra background. If you see students working on this question, make sure they can explain questions \#1 and \#2. If they
			cannot, send them back to those questions.
	\end{enumerate}

	\end{instructions}

\end{document}
