\subsection*{Learning Objectives}
Students need to be able to\ldots
\begin{itemize}
	\item Explain mathematical arguments using full sentences.
	\item Recognize when written work is unclear or incomplete.
	\item Produce executive summaries.
\end{itemize}

\subsection*{Context}
Students just had a midterm, so they may be mentally exhausted. However, their Homework \#3 is
due soon. We are working on setting expectations for the quality of students' written work. As well,
as part of Homework 3 and as part of their Final Report, they will be producing an executive summary.

This tutorial serves as practice for mathematical communication and authoring executive summaries.

\subsection*{What to Do}

Have students divide in groups as usual and start on question 1. Students do not have a concept of
what good mathematical communication is, so you will have to interrupt class to discuss this partway through the tutorial.

After most students have finished part 1 (a) and 1 (b), have a whole-class discussion about 1 (a) and 1 (b).
After getting student's opinions, announce the scoring set in 1 (c). Many students will be shocked. Get some ideas
from the students and make sure they idea of ``explanations should be written in full sentences'' comes out. Then,
let them continue on part (d).

Many students may spend the entire time on Question 1, but those that are on top of their mathematical communication
may work on Question 2. Question 2 is \emph{very} relevant for their homework. If they don't see this, ask them to read Homework 3 Question 3.

\subsection*{Wrap Up}

In all likelihood, 1 (d) will make a good wrap up. Have students come to the board during the last few minutes of work time and 
write up their ``fixes'' for Question 1. Then the wrap-up can be a whole-class discussion of whether they agree with these fixes, and
whether they could be further improved. Mention that there is a difference between ``showing your work'' and ``explaining''. While the 
example answers (A2) and (A3) may have shown their work, they did not explain anything!

\subsection*{Notes}
\begin{enumerate}
	\item Several of these are actual transcriptions of student work. Students believe it is okay to barf equations onto a page.
	This is what ``showing your work'' means to them. However, in this class, we do not ask for students to ``show their work''.
	We ask for them to \emph{explain}. This is different and not something they are used to.
	
	For parts (c), students might give you push back and demand that those answers deserve more part marks (you might even
	believe that some answers deserve more marks yourself\ldots). Do not get into an argument with the students about what 
	``fair'' marking is. Instead, explain that we are trying to communicate to you the explanations of MAT244, and how to achieve
	a good mark in \emph{this} class. Not some other class with a different marking scheme. 
	
	\item Students don't have experience with executive summaries and will be a lot more lost with this question since it's not
	clear what they should ``solve''.

	It is important that students recognize that the AI generated executive summary is \emph{bad}. While is passes the bar of having
	a visually appealing structure and being written in full sentences, it doesn't have the correct substance (this is vert common for AI
	generated work).
	
	Make sure students incorporate their answer to 2 (a) into their answer to 2 (b) part ii.
\end{enumerate}

%\paragraph{How to provide good feedback (15-20 min)}
%
%SPARK guidelines.
%
%Essentially, you will be replicating the workshop done in the head TA meetings this
%week.
%
%Start by asking students what they think makes a piece of feedback good or bad.
%They can either explain, or give examples. Connect these examples to the SPARK
%guidelines to explain what makes this feedback good or bad.
%
%If students don't provide any convenient examples, here are the examples of good/bad
%feedback we saw in this week's workshop:
%\begin{itemize}
%	\item ``Everything is perfect!''
%
%	\item ``Your grammar sucks. Did you even use spellcheck?!''
%
%	\item ``You should delete parts of this section from your modelling, you don't
%		need them.''
%	%		\item ``Your mind map has too many concepts in it. These are good ideas, but the mindmap should be readable without zooming in. I think you should remove at least five of them.''
%
%	\item ``This sentence is very long, which makes it hard to read; you could split
%		it into 3 sentences if you put periods in these places. If you need extra
%		help with writing, your college probably has a writing centre you can use.''
%
%	\item ``I thought it was insightful that you discussed the decline of the lobster
%		fishing industry in your hometown as a reason for exponential population decrease.
%		You've clearly shown your knowledge of your hometown's culture and how it
%		affects the population in mathematical terms.''
%\end{itemize}
%
%The SPARK guidelines are as follows:
%\begin{itemize}
%	\item \textbf{Specific:} Comments are linked to a discrete word, phrase, or
%		sentence.
%
%	\item \textbf{Prescriptive:} Prescriptive feedback offers a solution or
%		strategy to improve the work, including possible revisions or links to helpful
%		resources or examples.
%
%	\item \textbf{Actionable:} When the feedback is read, it leaves the peer knowing
%		what steps to take for improvement.
%
%	\item \textbf{Referenced:} The feedback directly references the task criteria,
%		requirements, or target skills.
%
%	\item \textbf{Kind:} It's mandatory that all comments be framed in a kind, supportive
%		way.
%\end{itemize}
%
%\subsection*{Notes}
%\begin{itemize}
%	\item Feedback doesn't need to be critical to be good; positive feedback can
%		also be useful in telling students what to continue doing, as long as it still
%		embodies the SPARK guidelines.
%
%	\item Encourage students to review the feedback that is given to them. If
%		students don't feel that the feedback they receive is useful, they should ask
%		their partner to elaborate.
%\end{itemize}